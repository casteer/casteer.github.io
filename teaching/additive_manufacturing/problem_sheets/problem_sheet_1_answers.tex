
%----------------------------------------------------------------------------------------
\documentclass{article}
\usepackage{exam}
\usepackage{graphicx}
\usepackage{capt-of}
\usepackage{hyperref}

%%%% If you are not using LaTex2e, replace the first two lines 
%%%% with the following:
%\documentstyle[exam]{article}
\paper{3D Manufacturing APH6010}
\degree{Applied Physics}
\subject{Introduction to Additive Manufacturing}
\title{Problem Sheet 1}
\note{Attempt all questions}
% \year{\number\thisyear}

% \version{*}
% \title{you must give the {\tt $\backslash$title} command}
% \paper{Thermal Physics : 3D Manufacturing APH6010}
% \semester{Spring Term}
% \campus{}

\graphicspath{{./images/}}

\newcommand\AM{additive manufacturing }

\begin{document}
\vspace{5mm}

\begin{center}
  \large{\bf SECTION A\\[5mm]
    Introduction to Additive Manufacturing}\\[20mm]
\end{center}


\begin{questions}
\item General aspects of \AM
\begin{questions}
 \item Define the terms additive and subtractive manufacturing, citing examples of both. 
 \\[0.2cm] \emph{Answer: Additive manufacturing produces parts or objects by adding material as the part is built; it only uses just enough material to build the part. An example of additive manufacturing is fused filament fabrication. Subtractive manufacturing starts with raw materials and removes material from this raw block of material. An example of subtractive manufacturing is the production of parts using a lathe - the raw block spins and tools remove material from it to produce the require part shape. }
 \item Give an application to which \AM is well-suited and stating your reasons why. 
 \\[0.2cm] \emph{Answer: Additive manufacturing is well-suited to any application that requires customisation and produces parts in low numbers. e.g. printing prostheses or dental implants. \AM cannot at present compete in cost with mass-production techniques. An interesting exception to this is when additive manufacturing is used to produce parts for another production process e.g. casts. }
 \item What are the implications of \AM for commercial manufacturing and industry? 
 \\[0.2cm] \emph{Answer: Digital transportation of goods to the consumer leading to cheaper supply costs and benefits for the environment. A more detailed discussion is given in Fabrication (the grey book).}
 \end{questions}
 \item Design and commercial implications of \AM
\begin{questions}
  \item Are you able to sell printed items from a design sharing website, such as Thingiverse, on Ebay? Why not?
  \\[0.2cm] \emph{Answer: It depends whether the license permits commercial use. Example licenses that permit this include a number of Creative Commons licenses (CC-BY, CC-BY-SA, CC-BY-ND...). Factors to consider when choosing a license include whether you wish derivatives to have the same licensing (CC-BY-SA), whether you would like no derivatives or not (CC-BY-ND), permit commercial purposes or not (CC-BY-NC).}
  \item You run a successful toy company that has for many years produced small superhero figurines which are well-liked by a community of fans. You see the rise of 3D printing as a threat to your business as now many people can afford to scan and reproduce your figures (without buying them from your company!). How would you protect your business from this loss of revenue? What would you do to engage with this new market of fans with 3D printers?
  \\[0.2cm] \emph{Answer: Potential options to engage the community include: produce official models with no-derivatives, no-commercial aspects of licensing, update figurines to have details finer than the typical feature resolution of printers, produce modular and official figurines that allow the community to rehash and design new figurines. If all fails, employ litigation.}
  \item As a hobbyist-designer, you would like the community to be free to use your design and all of its derivatives, attributing you appropriately, and for non-commercial use only. Which type of copyright license would you choose?
  \\[0.2cm] \emph{Answer: CC-BY-NC-SA - allows others to build upon the model non-commercially with suitable attribution to you.}
\end{questions}
\end{questions}

\begin{center}
  \large{\bf SECTION B\\[5mm]
    The Generalised AM Process}\\[20mm]
\end{center}

\begin{questions}
\item Describe the steps involves in producing a printed item. 
  \\[0.2cm] \emph{Answer: See the relevant chapter in Gibson}
\item What types of information do STL and g-code files hold? 
  \\[0.2cm] \emph{Answer: STL files contain the surface mesh of the model decomposed into a number of triangles. G-code files contain instructions to the printer for its set-up and movements. }
\item Explain why the ratio of the number of faces to the number of edges in a STL file is 1.5, if the mesh is watertight. 
  \\[0.2cm] \emph{Answer: Draw a triangular mesh and consider the three edges of a singular triangle. These edges are each shared between two triangular faces so there are 3/2 edges for the single triangular face; or mathematically, $F/E = 1.5$, where $F$ is the number of faces and $E$ is the number of edges.}
\item Annotate each line of the following g-code with the corresponding action by an fused filament fabrication printer. At what stage of printing do you think that this code will be executed. You may use a g-code ``cheatsheet''. 
\begin{verbatim}
	G28 X0 Y0  
	G1 Z150 F300
	M104 S0 
	M140 S0 
	M84 
\end{verbatim}
\emph{Answer: 
  	G28 X0 Y0  - Zero the axes\\
	G1 Z150 F300 - Move to Z=150 and set the feedrate to 1500mm/min (irrelevant)\\
	M104 S0 - Set the extruder temperature to zero\\
	M140 S0 - Set the bed temperature to zero\\
	M84 - Stop/Idle/Hold - stops everything. \\
This could be used at the end of a printrun.}
\end{questions}

% \begin{center}
%   \large{\bf SECTION C\\[5mm]
%     Polymer Physics}\\[20mm]
% \end{center}
% 
% \begin{questions}
% \item Definitions: 
% \begin{questions}
% \item What are monomers? 
% \item What is a polymer? 
% \item Describe the different categories of polymers and examples of each category.
% \item What is the difference between thermosoftening and thermosetting polymers? Which type is most appropriate to fused filament fabrication? 
% \item What order of magnitude in energy does it take to break a covalent bond? How does this compare to typical fluctuations in temperature? What are the implications for polymer motion? 
% \end{questions}
% \item Polymer materials properties
% \begin{questions}
%  \item Describe hydrogen, dipole-dipole, and van der Waals forces and order them in terms of their strength. 
%  \item Describe the effect of interchain forces on the mechanical properties of polymers. 
%  \item Define the elastic modulus. Sketch and describe the effect of temperature, throuh a glass transition, of a polymer's elastic modulus. 
%  \item Draw the structure of polylactide acid (PLA), its glass and melting transition temperatures. Is PLA a thermoplastic or thermosetting polymer? 
% \end{questions}
% \end{questions}

% \newpage
% \begin{center}
%   \large{\bf BACKGROUND READING }\\[20mm]
% \end{center}
% 
% \begin{itemize}
% \item Introductory \AM
% \begin{itemize}
% \item Chapter 1 in Additive Manufacturing Technologies: 3D Printing, Rapid Prototyping, and Direct Digital Manufacturing, Gibson. 
% \item Chapter 1 to 4 in Fabricated, H. Lipson and M. Kurman, Wiley. (2013).
% \end{itemize}
% \item The General Additive Manufacturing Process
% \begin{itemize}
% \item Chapters 1, 3, and 13 : Additive Manufacturing Technologies: 3D Printing, Rapid Prototyping, and Direct Digital Manufacturing, Gibson.
% \item Chapter 5 to 6 in Fabricated (1st Edition), H. Lipson and M. Kurman, Wiley. (2013). 
% \item Chapter 4 in Additive Manufacturing Technologies: 3D Printing, Rapid Prototyping, and Direct Digital Manufacturing, Gibson. 
% \end{itemize}
% \item Polymer Physics
% \begin{itemize}
%  \item Chapter 4 in Additive Manufacturing Technologies: 3D Printing, Rapid Prototyping, and Direct Digital Manufacturing, Gibson. 
% \end{itemize}
% \end{itemize}



\end{document}