%%%%%%%%%%%%%%%%%%%%%%%%%%%%%%%%%%%%%%%%%
% Tufte-Style Book (Documentation Template)
% LaTeX Template
% Version 1.0 (5/1/13)
%
% This template has been downloaded from:
% http://www.LaTeXTemplates.com
%
% Original author:
% The Tufte-LaTeX Developers (tufte-latex.googlecode.com)
%
% License:
% Apache License (Version 2.0)
%
% IMPORTANT NOTE:
% In addition to running BibTeX to compile the reference list from the .bib
% file, you will need to run MakeIndex to compile the index at the end of the
% document.
%
%%%%%%%%%%%%%%%%%%%%%%%%%%%%%%%%%%%%%%%%%

%----------------------------------------------------------------------------------------
%	PACKAGES AND OTHER DOCUMENT CONFIGURATIONS
%----------------------------------------------------------------------------------------

\documentclass[nols,a4paper,twoside,symmetric,justified,marginals=raggedouter]{tufte-book} % Use the tufte-book class which in turn uses the tufte-common class

\hypersetup{colorlinks} % Comment this line if you don't wish to have colored links
\usepackage{amsmath}
\usepackage{microtype} % Improves character and word spacing
\usepackage{graphicx}

\usepackage{listings}
\usepackage{lscape}
\usepackage{float}
\usepackage{longtable}

\lstset{basicstyle=\ttfamily}


\usepackage{lipsum} % Inserts dummy text

\usepackage{booktabs} % Better horizontal rules in tables

\usepackage{graphicx} % Needed to insert images into the document
\graphicspath{{graphics/}} % Sets the default location of pictures
\setkeys{Gin}{width=\linewidth,totalheight=\textheight,keepaspectratio} % Improves figure scaling

\usepackage{fancyvrb} % Allows customization of verbatim environments
\fvset{fontsize=\normalsize} % The font size of all verbatim text can be changed here

\newcommand{\hangp}[1]{\makebox[0pt][r]{(}#1\makebox[0pt][l]{)}} % New command to create parentheses around text in tables which take up no horizontal space - this improves column spacing
\newcommand{\hangstar}{\makebox[0pt][l]{*}} % New command to create asterisks in tables which take up no horizontal space - this improves column spacing

\usepackage{xspace} % Used for printing a trailing space better than using a tilde (~) using the \xspace command

\newcommand{\monthyear}{\ifcase\month\or January\or February\or March\or April\or May\or June\or July\or August\or September\or October\or November\or December\fi\space\number\year} % A command to print the current month and year

\newcommand{\openepigraph}[2]{ % This block sets up a command for printing an epigraph with 2 arguments - the quote and the author
\begin{fullwidth}
\sffamily\large
\begin{doublespace}
\noindent\allcaps{#1}\\ % The quote
\noindent\allcaps{#2} % The author
\end{doublespace}
\end{fullwidth}
}

\newcommand{\blankpage}{\newpage\hbox{}\thispagestyle{empty}\newpage} % Command to insert a blank page

\usepackage{units} % Used for printing standard units

\newcommand{\hlred}[1]{\textcolor{Maroon}{#1}} % Print text in maroon
\newcommand{\hangleft}[1]{\makebox[0pt][r]{#1}} % Used for printing commands in the index, moves the slash left so the command name aligns with the rest of the text in the index 
\newcommand{\hairsp}{\hspace{1pt}} % Command to print a very short space
\newcommand{\ie}{\textit{i.\hairsp{}e.}\xspace} % Command to print i.e.
\newcommand{\eg}{\textit{e.\hairsp{}g.}\xspace} % Command to print e.g.
\newcommand{\na}{\quad--} % Used in tables for N/A cells
\newcommand{\measure}[3]{#1/#2$\times$\unit[#3]{pc}} % Typesets the font size, leading, and measure in the form of: 10/12x26 pc.
\newcommand{\tuftebs}{\symbol{'134}} % Command to print a backslash in tt type in OT1/T1

\providecommand{\XeLaTeX}{X\lower.5ex\hbox{\kern-0.15em\reflectbox{E}}\kern-0.1em\LaTeX}
\newcommand{\tXeLaTeX}{\XeLaTeX\index{XeLaTeX@\protect\XeLaTeX}} % Command to print the XeLaTeX logo while simultaneously adding the position to the index

\newcommand{\doccmdnoindex}[2][]{\texttt{\tuftebs#2}} % Command to print a command in texttt with a backslash of tt type without inserting the command into the index

\newcommand{\doccmddef}[2][]{\hlred{\texttt{\tuftebs#2}}\label{cmd:#2}\ifthenelse{\isempty{#1}} % Command to define a command in red and add it to the index
{ % If no package is specified, add the command to the index
\index{#2 command@\protect\hangleft{\texttt{\tuftebs}}\texttt{#2}}% Command name
}
{ % If a package is also specified as a second argument, add the command and package to the index
\index{#2 command@\protect\hangleft{\texttt{\tuftebs}}\texttt{#2} (\texttt{#1} package)}% Command name
\index{#1 package@\texttt{#1} package}\index{packages!#1@\texttt{#1}}% Package name
}}

\newcommand{\doccmd}[2][]{% Command to define a command and add it to the index
\texttt{\tuftebs#2}%
\ifthenelse{\isempty{#1}}% If no package is specified, add the command to the index
{%
\index{#2 command@\protect\hangleft{\texttt{\tuftebs}}\texttt{#2}}% Command name
}
{%
\index{#2 command@\protect\hangleft{\texttt{\tuftebs}}\texttt{#2} (\texttt{#1} package)}% Command name
\index{#1 package@\texttt{#1} package}\index{packages!#1@\texttt{#1}}% Package name
}}

% A bunch of new commands to print commands, arguments, environments, classes, etc within the text using the correct formatting
\newcommand{\docopt}[1]{\ensuremath{\langle}\textrm{\textit{#1}}\ensuremath{\rangle}}
\newcommand{\docarg}[1]{\textrm{\textit{#1}}}
\newenvironment{docspec}{\begin{quotation}\ttfamily\parskip0pt\parindent0pt\ignorespaces}{\end{quotation}}
\newcommand{\docenv}[1]{\texttt{#1}\index{#1 environment@\texttt{#1} environment}\index{environments!#1@\texttt{#1}}}
\newcommand{\docenvdef}[1]{\hlred{\texttt{#1}}\label{env:#1}\index{#1 environment@\texttt{#1} environment}\index{environments!#1@\texttt{#1}}}
\newcommand{\docpkg}[1]{\texttt{#1}\index{#1 package@\texttt{#1} package}\index{packages!#1@\texttt{#1}}}
\newcommand{\doccls}[1]{\texttt{#1}}
\newcommand{\docclsopt}[1]{\texttt{#1}\index{#1 class option@\texttt{#1} class option}\index{class options!#1@\texttt{#1}}}
\newcommand{\docclsoptdef}[1]{\hlred{\texttt{#1}}\label{clsopt:#1}\index{#1 class option@\texttt{#1} class option}\index{class options!#1@\texttt{#1}}}
\newcommand{\docmsg}[2]{\bigskip\begin{fullwidth}\noindent\ttfamily#1\end{fullwidth}\medskip\par\noindent#2}
\newcommand{\docfilehook}[2]{\texttt{#1}\index{file hooks!#2}\index{#1@\texttt{#1}}}
\newcommand{\doccounter}[1]{\texttt{#1}\index{#1 counter@\texttt{#1} counter}}

\usepackage{makeidx} % Used to generate the index
\makeindex % Generate the index which is printed at the end of the document

% This block contains a number of shortcuts used throughout the book
\newcommand{\vdqi}{\textit{VDQI}\xspace}
\newcommand{\ei}{\textit{EI}\xspace}
\newcommand{\ve}{\textit{VE}\xspace}
\newcommand{\be}{\textit{BE}\xspace}
\newcommand{\VDQI}{\textit{The Visual Display of Quantitative Information}\xspace}
\newcommand{\EI}{\textit{Envisioning Information}\xspace}
\newcommand{\VE}{\textit{Visual Explanations}\xspace}
\newcommand{\BE}{\textit{Beautiful Evidence}\xspace}
\newcommand{\TL}{Tufte-\LaTeX\xspace}

\newcommand{\SL}{Stereolithography}

\usepackage{wrapfig}
\usepackage[framemethod=tikz]{mdframed}
\usepackage{sidecap}
\usepackage[dvipsnames]{xcolor}
\usepackage{rotating}
\usepackage{listings}
\usepackage{pdflscape}

\usepackage{capt-of}
\PassOptionsToPackage{hyphens}{url}\usepackage{hyperref}

\usepackage{tabularx}


% For the polymer figures 
\usepackage{chemfig}


%\includeonly{design}

%----------------------------------------------------------------------------------------
%	BOOK META-INFORMATION
%----------------------------------------------------------------------------------------

\title{Design Physics for \\% 
 Additive Manufacturing} % Title of the book

\author[C. A. Steer]{Chris Steer} % Author

%\publisher{Publisher of This Book} % Publisher

%----------------------------------------------------------------------------------------

\begin{document}

\frontmatter

%----------------------------------------------------------------------------------------
%	EPIGRAPH
%----------------------------------------------------------------------------------------

% \thispagestyle{empty}
% \openepigraph{The public is more familiar with bad design than good design. It is, in effect, conditioned to prefer bad design, because that is what it lives with. The new becomes threatening, the old reassuring.}{Paul Rand, {\itshape Design, Form, and Chaos}}
% \vfill
% \openepigraph{A designer knows that he has achieved perfection not when there is nothing left to add, but when there is nothing left to take away.}{Antoine de Saint-Exup\'{e}ry}
% \vfill
% \openepigraph{\ldots the designer of a new system must not only be the implementor and the first large-scale user; the designer should also write the first user manual\ldots If I had not participated fully in all these activities, literally hundreds of improvements would never have been made, because I would never have thought of them or perceived why they were important.}{Donald E. Knuth}

%----------------------------------------------------------------------------------------

\let\allcaps=\relax 
\maketitle % Print the title page

%----------------------------------------------------------------------------------------
%	COPYRIGHT PAGE
%----------------------------------------------------------------------------------------

\newpage
\begin{fullwidth}
~\vfill
\thispagestyle{empty}
\setlength{\parindent}{0pt}
\setlength{\parskip}{\baselineskip}
Copyright \copyright\ \the\year\ \thanklessauthor

\par\smallcaps{Published by \thanklesspublisher}

%\par\smallcaps{tufte-latex.googlecode.com}

\par Licensed under the Apache License, Version 2.0 (the ``License''); you may not use this file except in compliance with the License. You may obtain a copy of the License at \url{http://www.apache.org/licenses/LICENSE-2.0}. Unless required by applicable law or agreed to in writing, software distributed under the License is distributed on an \smallcaps{``AS IS'' BASIS, WITHOUT WARRANTIES OR CONDITIONS OF ANY KIND}, either express or implied. See the License for the specific language governing permissions and limitations under the License.\index{license}

\par\textit{First printing, \monthyear}
\end{fullwidth}

%----------------------------------------------------------------------------------------

\tableofcontents % Print the table of contents

%----------------------------------------------------------------------------------------

\listoffigures % Print a list of figures

%----------------------------------------------------------------------------------------

\listoftables % Print a list of tables

%----------------------------------------------------------------------------------------
%	DEDICATION PAGE
%----------------------------------------------------------------------------------------

\cleardoublepage
~\vfill
\begin{doublespace}
\noindent\fontsize{18}{22}\selectfont\itshape
\nohyphenation
Dedicated to Caroline for her love, support and encouragement.
\end{doublespace}
\vfill
\vfill

%----------------------------------------------------------------------------------------
%	INTRODUCTION
%----------------------------------------------------------------------------------------

\cleardoublepage
\chapter*{Introduction} % The asterisk leaves out this chapter from the table of contents

% This sample book discusses the design of Edward Tufte's books\cite{Tufte2001,Tufte1990,Tufte1997,Tufte2006} and the use of the \doccls{tufte-book} and \doccls{tufte-handout} document classes.

During late 2015 and early 2016, I developed a course which focussed on the underlying physics of additive manufacturing for the Applied Physics degree at St Mary's University, Twickenham. The course was well-received but time was against me and I ended up with a number of hand-written notes and sheets. The notes lay on my desk for a while and, after some pestering to organise and file these, I eventually got around to writing these up, and this is the end product of this effort. 

So why should Physicists get interested in additive manufacturing? Partly the truth is that it is an area of strongly applied Physics which is evidenced by the rest of this book; partly, in areas such as this, there is much overlap between engineering and physics and it's difficult to say who (Physicist or Engineer) is best-placed to study this; and partly it's my experience that through understanding the underlying principles, the Physicist will be able to contribute to the community. To be honest, mostly it's enjoyable.  

As I developed the course, it became more obvious of the areas of overlap between Physics and Engineering, it's seem highly appropriate that it lies within a broader Applied Physics course. 

%----------------------------------------------------------------------------------------
%	CHAPTER 1
%----------------------------------------------------------------------------------------

\chapter{Additive Manufacturing, Intellectual Property and Open Source Design}
\label{ch:society}

% \documentclass[a4paper,10pt]{article}
% \usepackage[utf8]{inputenc}
% 
% %opening
% \title{Social Impact of Additive Manufacturing}
% \author{C. A. Steer}
% 
% \begin{document}
% 
% \maketitle
% 
% \begin{abstract}
% \end{abstract}

% http://www.theatlantic.com/magazine/archive/2000/03/the-kept-university/306629/

\section{Intellectual property and AM}

Intellectual property \sidenote{For the ideallist, this is the oddest chapter - an aspect of 3D printing design that I hope that you'll never have to use regularly and is dry in its pure lack of physics. \emph{But}, if you'd like to be credited in a particular manner or would like to know how cite your new friends in the 3D printing community properly (so that they become lifelong friends), then you need to know about the basics of intellectual property. Don't take too long to read this chapter, skim it and know that it's here to help but generally I'd move on to the good stuff elsewhere in this book.} comprises the creative and innovative inventions of designers or technologists and important to understand regardless of whether you are designing for personal or commercial reasons. 

When designing for personal enjoyment you may wish to provide your design back to the community and, especially if it's based on someone else's design, then you need to be aware of both the right way to credit your inspiration, or even how you'd like to cited, and the implications of their or your own choice of licensing\sidenote{Thingiverse, see \url{www.thingiverse.com}, is a popular website to upload your 3D printed designs and deals with intellectual property in an interesting manner. The user has a selection of licenses to choose from, including a variety of Creative Commons licenses, giving details on how you'd like to cited, or have strange acronyms such as BSD, GPL or LGPL. Please see section \ref{licensing} for more details on these.}

\subsection{Copyright Protection}

Copyright, a form of intellectual property, refers to ideas within a medium so, for example, this could be literary, musical or another artistic endeavour. To copyright a item, the designer can place a copyright notice on the item such as `Copyright 2016 \copyright Joe Bloggs`, or even leave it off as copyright is applied from the moment that the thing is created\sidenote{In the UK, a good website to start researching into copyright is: \url{https://www.gov.uk/copyright/overview}}. Copyright means that others cannot copy or distribute your work. 

\subsection{Patent Protection}

Patents are another form of intellectual property in which the invention or idea has technical novelty, and is stated in a number of claims within the formal patent documentation. Patents protect the inventor's idea for a short length of time and allow the inventor time to develop and commercially exploit their ideas. During this time, no one else is able to use this idea without the patent-owner selling the rights. A patent remains in force for 20 years and elsewhere at least 20 years. 

\subsection{Trademark Protection}

Trade marks are a logo or symbol that can be licensed to third-parties. The main property which distinguishes it from copyright is that a trademark can be protected as long as it's in use. The trademark may not necessarily include an inventive step but does need to be unique and recognisable of the brand. Within the creative industries, for example, the film or book may be protected by copyright and the characters trademarked - this allows third party companies to 

If you're designing for pleasure or personal study, and don't intend to commercialise your ideas, then generally you don't need to worry about infringing other's IP (by this, I mean that there is an established legal defence [citation]). It's still important to be aware of the potential areas of patent or copyright infringment which, as examples, might include mechanisms for locking jars\sidenote[][-8\baselineskip]{Let's say, for example, that with your trusty dual extrusion FDM machine you designed you an all-in-one cylindrical enclosure with a clamping top onto a flexible gasket, intended for food and wanted to sell your design for others to buy. It's important to know what else has been protected. The Kilner jar is a famous example of sealing mechanism. The original jar was a screw-top. Later designs comprise a hinged lid and clamp which press the lid down on a deformable rubber seal. The clamp's downward pressure provides a seal enabling food to be preserved for longer. The design was patented in 1858. Kilner jars are, however, trademarked. http://www.mylearning.org/inventors-and-inventions-from-yorkshire/p-2619/}, or reproduction of film mechandise (although this is almost definitely a copyright infringement). 

% \subsection{Principles of...}

%New scissors patent
%http://www.boldip.com/bold-move-room-for-improvement-on-even-the-most-common-mechanical-devices-scissors/


%http://3dprintingindustry.com/news/many-3d-printing-patents-expiring-soon-heres-round-overview-21708/

%http://united-kingdom.taylorwessing.com/download/article_3d_printer_guide.html#.V3dj_e2VvCI
% \section{Intellectual Property Maneuvers: The Rise of Open Source Additive Manufacturing}
% 
% \subsection{1980s - Come on Eileen}
% 
% The first recognised 3D printing technique was Stereolithography, developed in 1984 by Charles Hull who later founded 3D Systems. This is discussed in more detail later on in this book. After patenting the technique, Charles Hull founded 3D Systems Corporation and started to commercialise this in the SLA-1 system in 1988. 
% 
% Selective laser sintering was also developed around this time (1987) by a Carl Deckard of the University of Texas. The patent for SLS was issued in 1989 and licensed to DTM Inc. which as a company was bought by 3D Systems. 
% 
% Stratasys was another rapid prototyping company around at the time. Whilst the Berlin wall fell in 1989, Scott Crump also developed the fused deposition modelling technique, the basis for much of the open source printing techniques. At the same time, EOS GmbH was also founded in Germany and became well-respected in the area of laser sintering.
% 
% \subsection{1990s - Nothing Compares 2 U}
% 
% Whereas the 1980s a few techniques were invented, in the 1990s these became more established and were joined by others. The FDM patent was awarded to Stratasys in 1992.  
% 
% Commercial Stereolithography machines started to become available in the 1990s, 
% 
% During the late twentieth and early twenty first century, with the development of the internet, software development became a collaborative movement. Software development moved from private enterprise developing closed source code repositories to the open source. This enabled software engineers to engage others who were similarly interested. 
% 
% Additive Manufacturing was developed alongside the development of the free software foundation, that later turned into the open source movement. However, as required by this intellectual protection, AM was developed solely by a small number of innovative companies. 
% 
% 3D Systems Inc. 
% 
% Stratasys
% 
% \subsection{2000s - Wannabe}
% 
% In the mid 2000s, the RepRap project came about, following funding of Adrian Bowyer at Bath University by the UK Engineering and Physical Sciences Research Council. 

\section{Open Source and Collaborative - a new way of working?}




Cathedral and the Bazaar 

\section{Open Source Licensing}
\label{licensing}

`A free software license is a notice that grants the recipient extensive rights to modify and redistribute that software. These actions are usually prohibited by copyright law, but the rights-holder (usually the author) of a piece of software can remove these restrictions by accompanying the software with a software license which grants the recipient these rights.'

Creative Commons (CC) is an organisation that provides a legal framework for the creative sharing of knowledge. The designer chooses the features of a CC license that they would like and then is free to share it on the appropriate website of their choice. There are important considerations which include
\begin{itemize}
 \item Would you like the design to be reused for commercial purposes? 
 \item Are derivative, modified variants of the designs allowed?
 \item Should derivative works be licensed under the same conditions?
\end{itemize}

Open to closed licensing
\begin{itemize}
 \item Public Domain: This mark or license allows the designer to say that this work is free from any form of copyright. Typically very old creative works are licensed in this way.  
 \item CC0: The CC0 mark is used with works that the designer would like no restrictions on its sharing, and that the works are younger than national copyright restrictions allow. 
 \item CC-BY: This license allows the user to redistribute the design for any purpose (even commercially), and to modify it in any way. Reproduction must also include attribution to the original designer and indicate how it was modified.  \url{https://creativecommons.org/licenses/by/2.0/}
 \item CC-BY-SA: This license follows the same conditions as CC-BY and the derivative work must also be under the same license. 
 \item CC-BY-ND: This license follows the same conditions as CC-BY and does not permit derivative works without further permission. 
 \item CC-BY-NC: This license follows the same conditions as CC-BY and permits any other usage except for commercial applications. 
 \item CC-BY-NC-SA: This license combines CC-BY with the same non-commercial and shearealike restrictions above. 
 \item CC-BY-NC-ND: This license combines CC-BY with the same non-commercial and no derivatives restrictions above. 
\end{itemize}

\section{Open Source Issues}

\section{Further Reading}

\begin{itemize}
 \item Fabricated: The New World of 3D Printing, by Hod Lipson and Melba Kurmann
\end{itemize}


% \end{document}


%----------------------------------------------------------------------------------------


%----------------------------------------------------------------------------------------
%	CHAPTER 2
%----------------------------------------------------------------------------------------

\chapter{Introduction to 3D Printing Techniques and Technologies}
\label{ch:intro_tech}



\chapter{Overview}




\section{Print properties}

\section{Printing in layers}

Additive manufacturing is a term developed later on in the development of 3D printing to describe the common property of adding materials to the part being produced.Adding materials in layers results in new possible stuctures as the designer has access to the interior of the model during production. This is not usually the case for conventional manufacturing means and opens up the possibility of printing many different types of joints and motion from a single piece. 

Two examples are shown in Figures \ref{fig:quad_bearings} and \ref{fig:articulated_elephant} in which joints giving a spinning motion, for the nested bearings, or articulation for the elephant toy's legs and trunk are printable. 

\begin{figure}[h]
 \centering
 \includegraphics[width=0.8\textwidth]{./pictures/Quad-Bearings.png}
 % Quad-Bearings.png: 0x0 pixel, 300dpi, 0.00x0.00 cm, bb=
 \caption{An illustration of the ability of 3D printed items to produced spinning joints. After printing the joints must be worked so that the item is spinning freely. \url{http://www.thingiverse.com/thing:1478386}. }
 \label{fig:quad_bearings}
\end{figure}


\begin{figure}[h]
 \centering
 \includegraphics{./pictures/elephant_articulated.png}
 % elephant_articulated.png: 0x0 pixel, 300dpi, 0.00x0.00 cm, bb=
 \caption{An elephant with articulated legs illustrating a slightly different joint type \url{http://www.thingiverse.com/thing:257911}. }
 \label{fig:articulated_elephant}
\end{figure}

The only constraint is that the joint must somehow be attached to the previous layer, even if it is by a small area. When the part has finished printing, it is lifted from the bed and the joint freed of the attachment.

There are numerous benefits of AM techniques: There is little or no material or energy wastage, especially compared to conventional means, as only the necessary material is added; new structures can be produced given tha the techniques 

\textbf{Producing parts by successively adding new material, typically in layers, is a characteristic that all AM techniques share.}

\subsection{Alternative Names for the Additive Manufacturing}

Whilst grappling to describe the new and emerging process of AM, various terminology has been so it's important to be aware that alternative terminology may be used including: 
\begin{itemize}
 \item Additive manufacturing
 \item (Solid) Freeform fabrication 
 \item Automated fabrication 
 \item 3D printing - this was originally a specific technique but has since been come to mean the whole community and category of techniques, at least colloquially. 
 \item Stereolithography - this is a specific, and the original, AM technique and not the class of techniques 
 \item Rapid prototyping 
\end{itemize}


\section{The Taxonomy of 3D Printing Techniques}

There are many additive manufacturing techniques which are summarised in Figure \ref{fig:am_overviews}. These can be categorised as exploiting photopolymerisation, material extrusion, jetting, powder fusion, direct energy deposition and sheet lamination. The most well-known techniques are fused deposition modelling (also known as fused filament fabrication, or FFF) and stereolithography. 

\begin{sidewaysfigure}[ht!!]
 \centering
 \includegraphics[width=\textwidth]{./graphics/additive-manufacturing-infographic_r10-06.png}
 \caption{An overview of different techniques and manufacturers, provided by 3DHubs.}
 \label{fig:am_overviews}
\end{sidewaysfigure}


\subsection{Solidification Processes}

Local solidification of the material feedstock is a key aspect of AM techniques and various approaches have been taken. These solidification processes include: 
\begin{itemize}
 \item \textbf{Curing}: The solidification of ultraviolet-sensitive polymer resin which is initially fluid and a curing, or photopolymerisation, process is induced by focussing a laser spot on the resin. The curing process causes the resin to solidify locally where the laser interacts with the resin. 
 \item \textbf{Jetting/Extrusion}: Extrusion is the process of heating material to a point where it becomes fluid, it is deposited on a print surface, cools and solidifies. Jetting is a very similar process that may involve a chemical process and may not necessary involve heat 
 \item \textbf{Sintering}: The density of powders can be greatly increased by heating, promoting the diffusion of atoms and the growth in size of powder grains. Heating can be applied by a high-powered laser in order to solidify a part in a layer-by-layer manner. 
 \item \textbf{Binding}: Binding is when an epoxy or glue is introduced to a powder bed in areas of a layer where solidification is required. 
 \item \textbf{Lamination}: Lamination is the process of placing sheets of material down and cutting them to form the part. 
\end{itemize}

  
\subsection{Material Feedstock Types }

In all of additive manufacturing techniques, there must be some way of introducing material to the print volume where it is needed. There are a number of options 
\begin{itemize}
 \item \textbf{Powder}: Powder is one feedstock and can be used as a bed, with suitable binder, melting, or sintering means to produce the local solid region in the print item. 
  \item \textbf{Filament}: Pushing filament into a means of transforming it into a solid in the print area is also quite common. This has been used with melting of plastic and welding of metal filaments. 
  \item \textbf{Photosensitive resin}: Photopolymer resin can be made to solidify under the action of an ultraviolet or blue laser. 
  \item \textbf{Sheet}: 
\end{itemize}

\subsection{Categories of Additive Manufacturing Techniques}

Given the combination of solidifiation process and material feedstock, it is possible to now classify the many techniques that are available to the desginer: 

\textbf{Vat Photopolymerization}

Photosensitive resin seems a miraculous material - viscous liquid when unreacted and solid when not. By combining a resin vat with a laser or ultraviolet light to draw the solid parts of each layer, these techniques can produce very high resolution parts in the solid photopolymer plastic resin. 

\textbf{Material Extrusion}

Fused deposition modelling, also known as fused filament fabrication, is a very common method that is often the entry level technique for many newcomers to the area. A filament is pushed by a stepper motor into a hot region in the extruder, which then melts and successive filament increases the pressure, pushing it out of the small diameter hole in the extruder. As it leaves the nozzle, the filament is placed on either the printbed or a previous layer where it bonds and solidifies as it cools. The area of extrusion prints plastics and plastics doped with other materials, such as metal powder or advanced materials such as graphene. 

\textbf{Material Jetting}

Material jetting introduces the material onto the print surface and immediately solidifies using an suitable method. The jetted material can vary quite alot from plastic, metal and wax with UV light, heat curing, and cooling solidification methods 

\textbf{Powder Bed Fusion}

Powder bed fusion techniques place smoothed powder layers 

\textbf{Direct Energy Deposition}

\textbf{Sheet Lamination}





\subsection{Time taken for passing a printhead over a layer} 

As the main characteristic of additive manufacturing is the layered production of objects, we can consider the simplest situation when the printer is covering a rectangular region in the $x$ and $y$ directions with a length $L$ and width $W$ in these respective directions. Let the speed of the print head be $v_{x}$, the suffix indicates than the printhead mostly in the $x$ direction. 

The time taken for one line in the $x$ direction is then $\frac{L}{v_{x}}$, the printhead then move one linewidth in a vertical direction and moves in the opposite direction of $x$. The time taken to step a linewidth along $y$ is $\frac{L_{w}}{v_{x}}$, where we assume that the printhead moves at a constant speed. 

The number of lines in $x$ is $N_{x} = \frac{W}{L_{w}}$. 

The total time taken to cure the rectangular region is 
\begin{equation}
  t = \frac{W}{v_{x} L_{w}} \left( L + L_{w} \right). 
\end{equation}

For a general and fully-covered region of area $A$, the time taken will be of the order of $\sim \frac{A}{v_{x} L_{w}}$. The time to cover an area fully varies inversely with the linewidth, indicating one of the main trade-offs with additive manufacturing - namely that finer prints take longer. 

\section{Fill-in for interior regions}

[Types of filling in and the choice of one or the other]

\subsection{Hexagonal fill}

\subsection{Rectilinear fill}

\subsection{Fused Deposition Modelling Overview}


\begin{center} 
\begin{figure}[h!!]
 \centering
 \includegraphics[width=0.8\textwidth]{./graphics/process_fdm.png}
 % process_fdm.png: 0x0 pixel, 300dpi, 0.00x0.00 cm, bb=
 \caption{A schematic of the fused deposition modelling process.}
\end{figure}
\end{center}

% \subsection{Fused Deposition Modelling Overview}
% 
% \begin{figure}[h]
%  \centering
%  \includegraphics[width=0.8\textwidth]{./graphics/process_lom.png}
%  % process_lom.png: 0x0 pixel, 300dpi, 0.00x0.00 cm, bb=
%  \caption{A schematic of the fused deposition modelling process.}
% \end{figure}

\subsection{Stereolithography Overview}

\begin{figure}[h!!]
 \centering
 \includegraphics[width=0.8\textwidth]{./graphics/process_sl.png}
 \caption{A schematic of the stereolithography additive manufacturing process.}
\end{figure}


% \begin{center} 
% \begin{figure}[h]
%  \centering
%  \includegraphics[width=0.8\textwidth]{./graphics/process_dmls.png}
%  % process_dmls.png: 0x0 pixel, 300dpi, 0.00x0.00 cm, bb=
%  \caption{An illustrative Direct metal laser sintering}
% \end{figure}
% \end{center}



\chapter{Design Process}
\label{ch:design}
\section{The General Additive Manufacturing Design Flow}


Although there are a variety of different techiques, the steps involved in producing parts can be generalised and are common to all. 

\begin{enumerate}
 \item  \textbf{Computer-aided-design (CAD)}: CAD software is available as a tool for the designer, engineer or physicist and used to draw the 3D model of the part. 
 \item  \textbf{Export model surface data file}: This step exports the 3D model's surface data in a file for slicing
 \item  \textbf{Slicing and g-code export}: This utilises the parameters of the machine set-up in order to slice the model. This step also provides the path of the printhead for each slice and any other necessary commands, such as modifying the printhead speed or temperature of any heaters. This step may also introduce structures to aid the job which may include removable supports. 
 \item  \textbf{Machine set-up}: Before printing, the machine will need to be correctly set-up. This may include the levelling of the bed, homing of the printhead, or warming up heaters. 
 \item \textbf{Print}: This step is when the machine is used to print the object. Care ought to be taken as environmental conditions, incorrectly set-up machine, or an accident may result in a failed print. 
 \item \textbf{Part removal and clean-up} : The part must be separated from the bed and cleaned up ready for the application. This could include removing any unwanted structures from the part, or chemical processing or sanding to smooth surfaces down. 
\end{enumerate}

\newpage
\section{Step 1 : Computer aided part design for Additive Manufacturing}

\subsection{Design for Additive Manufacturing}

Parts for additive manufacturing can be designed quite easily by the newcomer. There are several important areas to think about: 
\begin{itemize}
 \item Is the design printable? There are limitations to printing design on 3D printers and its important to be mindful of these. These limitations include issues such as size (is the object too big?), time to print (how long will it take to print? And is the printer ok to leave for a long time?), hard-to-print design features (are there any overhangs? Or does the object have a small based in contact with the print bed for FFF?).
 \item What does each feature of the design need to do? Does it do this? For example, here we could be thinking about 3D printed shoes; in which case, quick calculations of the weight and load on the shoes and whether the shoe can do this would be helpful. 
\end{itemize}

If you need to communicate the design to someone else, it's important to think about: 
\begin{itemize}
 \item Have you included all the dimensions you need to? 
 \item What are the units of the dimensions? 
 \item Which material is it to be printed from? 
 \item How does this part fit to another or an assembly? 
\end{itemize}

Many CAD packages are available including: 
\begin{itemize}
 \item OpenSCAD : Parametric modelling tool (developed by 3D printing community)
 \item Solidworks : Used by design engineers, GUI-based
 \item SketchUp :  Also used within community, GUI-based
 \item FreeCAD : Open source GUI-based CAD package 
 \item Rhino : Design modelling tool  
\end{itemize}

There are also online browser based software : www.tinkercad.com, www.3dtin.com, www.onshape.com


\subsection{Programmatic CAD using OpenSCAD}

There are many options to use to develop your designs. As a Physicist, I'm more comfortable with code than graphical user interfaces and actually prefer the ability to modularise the code. Along this vein, OpenSCAD is a very popular tool with the community that has also been extended with a number of libraries to provide, for example, screw threads or bevels and so on. 

The OpenSCAD cheat sheet remains the most immediate source of syntax for the intermediate user. 

The best way to learn OpenSCAD is by writing code and experimenting with the different commands. Here I take two approaches, the first is through a number of worked examples and I include a number of worksheets for the interested student. 

\subsection{Creating a simple box}

The first thing to note is that OpenSCAD doesn't necessarily specify its units. However, when they are translated to a surface file format, such as STL, and then to g-code, commands that instruct the nozzle to move, the units are assumed to be in millimeters\marginnote{In fact, the majority of engineers assume that dimensions are in millimeters.}. In short, assume that the units are in millimeters and degrees. 

So each command, that produces a shape, in OpenSCAD produces a solid shape. Let's produce a simple cube shape
\begin{lstlisting}
cube(100,center=true);
\end{lstlisting}
which is a cube with a side length of 100mm, centred at the origin. So now we'd like to make an empty box which is really a difference two shapes: A cuboid to describe the outer surface and a smaller one to describe the inner surface. 
\begin{lstlisting}
difference(){
  cube(100,center=true);
  cube(95,center=true);
  }
\end{lstlisting}
which will create a hollow cube with a wall thickness of 5mm (the difference of the first 100mm cube and the second 95mm cube). However now, if this is rendered, the box will still look to be solid as we need to move the inner cube up to make an opening. 
\begin{lstlisting}
difference(){
  cube(100,center=true);
  translate([0,0,10])
  cube(95,center=true);
  }
\end{lstlisting}
So now the inner cube is centred on the point $\left(0,0,\textrm{10~mm}\right)$. The box has an opening and wall thicknesses of 5mm everywhere except for the base of the box, for which the wall thickness is 15mm (5mm plus the 10mm due to the translation). The makes the box feel very bottom-heavy so may not be desireable. To make the wall thickness 5mm everywhere we need to elongate the inner cube in the vertical direction. 
\begin{lstlisting}
difference(){
  cube(100,center=true);
  translate([0,0,10])
  cube([95,95,110],center=true);
  }
\end{lstlisting}
In order to make the lower wall have a thickness of 5mm, it must be at $\left(0,0,\textrm{-45~mm}\right)$. The inner cuboid is translated upwards by 10mm so before the translation the lower face must be at -55mm. Therefore we need a cuboid which is 110mm high. 







\newpage
\section{Step 2 : Exporting Surface Files and their Formats}

\subsection{Stereolithography Tesselation Language (STL) file format}

STL files encode the surface of the part as a number of triangles, which are also called facets. A surface is comprised of many facets sharing their edges and making the whole part watertight; watertightness mean that there are no gaps in the mesh through which fictional water may escape from. 

As the triangular faceted surface is an approximation of many smooth surfaces, there is a trade-off between the accuracy of a model and the size of the file. For example, a low resolution model of the famous Stanford Bunny model\sidenote{The Stanford Bunny is so famous in computer vision circles that it has its own Wikipedia page \url{https://en.wikipedia.org/wiki/Stanford_bunny}. There are a number of other models which are listed at \url{https://en.wikipedia.org/wiki/List_of_common_3D_test_models}.} comprising 33,000 vertices is around 3MB. A high resolution version of 135,000 vertices is around 13MB. 

\begin{figure}[h!]
 \centering
 \includegraphics[width=0.8\textwidth]{./pictures/bunnies.png}
 % bunnies.png: 0x0 pixel, 300dpi, 0.00x0.00 cm, bb=
 \caption{Four different triangular representations of the Stanford Bunny model with increasing number of triangles from top-left to bottom-right.}
 \label{fig:bunnies}
\end{figure}


STL files can be binary or ascii and the typical file contain many facets. An example STL file for a unit cube, produced using OpenSCAD, is shown in Table \ref{stl}. The STL listing shows the syntax: a model is contained between the commands\sidenote{n.b. [name] is a label for the model in the STL file.} ``solid [name]'' and then ``endsolid [name]''. Each triangular facet is then between ``facet normal [vector]'' and\sidenote{n.b. [vector] is the components of the normal vector to the facet.} ``endfacet'', and the vertices coordinates are contained with ``outer loop'' and ``endloop''. Each triangle vertex has the format ``vertex [coordinates]''\sidenote{n.b. Replace [coordinates] with the three coordinate of each vertex.}. 

\begin{figure}[ht!]
 \centering
 \includegraphics[width=0.8\textwidth]{./pictures/unitcube.png}
 % unitcube.png: 0x0 pixel, 300dpi, 0.00x0.00 cm, bb=
 \caption{The unit cube model and lines indicating facets. This model corresponds to the STL listing in Table \ref{stl}.}
 \label{fig:unitcube}
\end{figure}


[Exercise to take the rabbit model and reduce the number of triangles with meshcad]

% Struggle to get this on a single page with tiny font
\begin{margintable}
\begin{minipage}{\textwidth}
\begin{lstlisting}[lineskip=-4.5pt,basicstyle=\tiny]
solid OpenSCAD_Model
  facet normal -0 0 1
    outer loop
      vertex 0 1 1
      vertex 1 0 1
      vertex 1 1 1
    endloop
  endfacet
  facet normal 0 0 1
    outer loop
      vertex 1 0 1
      vertex 0 1 1
      vertex 0 0 1
    endloop
  endfacet
  facet normal 0 0 -1
    outer loop
      vertex 0 0 0
      vertex 1 1 0
      vertex 1 0 0
    endloop
  endfacet
  facet normal -0 0 -1
    outer loop
      vertex 1 1 0
      vertex 0 0 0
      vertex 0 1 0
    endloop
  endfacet
  facet normal 0 -1 0
    outer loop
      vertex 0 0 0
      vertex 1 0 1
      vertex 0 0 1
    endloop
  endfacet
  facet normal 0 -1 -0
    outer loop
      vertex 1 0 1
      vertex 0 0 0
      vertex 1 0 0
    endloop
  endfacet
  facet normal 1 -0 0
    outer loop
      vertex 1 0 1
      vertex 1 1 0
      vertex 1 1 1
    endloop
  endfacet
  facet normal 1 0 0
    outer loop
      vertex 1 1 0
      vertex 1 0 1
      vertex 1 0 0
    endloop
  endfacet
  facet normal 0 1 -0
    outer loop
      vertex 1 1 0
      vertex 0 1 1
      vertex 1 1 1
    endloop
  endfacet
  facet normal 0 1 0
    outer loop
      vertex 0 1 1
      vertex 1 1 0
      vertex 0 1 0
    endloop
  endfacet
  facet normal -1 0 0
    outer loop
      vertex 0 0 0
      vertex 0 1 1
      vertex 0 1 0
    endloop
  endfacet
  facet normal -1 -0 0
    outer loop
      vertex 0 1 1
      vertex 0 0 0
      vertex 0 0 1
    endloop
  endfacet
endsolid OpenSCAD_Model
\end{lstlisting}
\end{minipage}
\caption{The STL listing for a unit cube, produced using OpenSCAD.}
\label{stl}
\end{margintable}


\subsection{Additive Manufacturing File Format (AMF)}

The AMF file format is a newer and more complex object description language which allows the designer to include material and other metadata with the surface file. AMF is based on the extensible markup language (XML) in which sections are book-ended by \begin{verbatim} <tag> \end{verbatim} and \begin{verbatim} </tag> \end{verbatim} pairs. 

In AMF files, the tag \begin{verbatim} <object> \end{verbatim} starts an object description 

\newpage
\begin{margintable}
\begin{minipage}{\textwidth}
\begin{lstlisting}[lineskip=-4.5pt,basicstyle=\tiny]
<?xml version="1.0" encoding="UTF-8"?>
<amf unit="millimeter">
 <metadata type="producer">OpenSCAD 2015.03-1</metadata>
 <object id="0">
  <mesh>
   <vertices>
    <vertex><coordinates>
     <x>-0.5</x>
     <y>-0.5</y>
     <z>-0.5</z>
    </coordinates></vertex>
    <vertex><coordinates>
     <x>0.5</x>
     <y>-0.5</y>
     <z>-0.5</z>
    </coordinates></vertex>
    <vertex><coordinates>
     <x>0.5</x>
     <y>-0.5</y>
     <z>0.5</z>
    </coordinates></vertex>
    <vertex><coordinates>
     <x>-0.5</x>
     <y>-0.5</y>
     <z>0.5</z>
    </coordinates></vertex>
    <vertex><coordinates>
     <x>-0.5</x>
     <y>0.5</y>
     <z>0.5</z>
    </coordinates></vertex>
    <vertex><coordinates>
     <x>-0.5</x>
     <y>0.5</y>
     <z>-0.5</z>
    </coordinates></vertex>
    <vertex><coordinates>
     <x>0.5</x>
     <y>0.5</y>
     <z>0.5</z>
    </coordinates></vertex>
    <vertex><coordinates>
     <x>0.5</x>
     <y>0.5</y>
     <z>-0.5</z>
    </coordinates></vertex>
   </vertices>
   <volume>
    <triangle>
     <v1>0</v1>
     <v2>1</v2>
     <v3>2</v3>
    </triangle>
    <triangle>
     <v1>3</v1>
     <v2>0</v2>
     <v3>2</v3>
    </triangle>
    <triangle>
     <v1>3</v1>
     <v2>4</v2>
     <v3>0</v3>
    </triangle>
    <triangle>
     <v1>0</v1>
     <v2>4</v2>
     <v3>5</v3>
    </triangle>
    <triangle>
     <v1>3</v1>
     <v2>2</v2>
     <v3>6</v3>
    </triangle>
    <triangle>
     <v1>4</v1>
     <v2>3</v2>
     <v3>6</v3>
    </triangle>
    <triangle>
     <v1>5</v1>
     <v2>7</v2>
     <v3>0</v3>
    </triangle>
    <triangle>
     <v1>0</v1>
     <v2>7</v2>
     <v3>1</v3>
    </triangle>
    <triangle>
     <v1>1</v1>
     <v2>7</v2>
     <v3>6</v3>
    </triangle>
    <triangle>
     <v1>2</v1>
     <v2>1</v2>
     <v3>6</v3>
    </triangle>
    <triangle>
     <v1>4</v1>
     <v2>6</v2>
     <v3>5</v3>
    </triangle>
    <triangle>
     <v1>5</v1>
     <v2>6</v2>
     <v3>7</v3>
    </triangle>
   </volume>
  </mesh>
 </object>
</amf>
\end{lstlisting}
\end{minipage}
\caption{The AMF format listing for a unit cube, produced using OpenSCAD.}
\label{stl}
\end{margintable}



\newpage
\section{Step 3 : Slicing the CAD Model}

\subsection{Software}

There are several software programs that can slice a CAD model and produce g-code, describing the motion of the printhead in each layer. 
These include but more may be available by the time you read this:  
\begin{itemize}
 \item Cura : \url{https://ultimaker.com/en/products/cura-software}
 \item Craftware : \url{https://craftunique.com/craftware}
 \item Simplify3D : \url{https://www.simplify3d.com/}
 \item Slic3r : \url{http://slic3r.org/}
 \item IceSL : \url{https://members.loria.fr/Sylvain.Lefebvre/icesl/}
\end{itemize}
A nice summary is also available at \url{https://all3dp.com/best-3d-printing-software-tools/}

\subsection{Glossary of slicing parameters}

The focus here is on parameters used in fused deposition modelling: 

[How do these differ for other techniques?]

\textbf{Layer Height :} As the name suggested, the layer height is the distance that the printhead moves vertically away from thesurface for each layer. The print quality can be quite sensitive to the layer height: If the layer height is set too high and the filament as it exits the nozzle will cool and not stick to the previous layer. If the layer height is too low then will ooze out of the side of the nozzle as it is pushed on to the previous layer, which will mean that the exterior of the part and any surfaces become rough\sidenote{There is a nice layer height calculator at \url{http://prusaprinters.org/calculator/} , which ensures that the layer height is a integer multiple of the step from the motors.}.

\textbf{Shell Thickness :} The shell thickness is the number of filament roads around the edge of the model in each slice. The shell thickness ought to be close to being a multiple of the road thickness and close to the nozzle diameter. 

\textbf{Infill Pattern:} For the interior regions of the part, most of the time it does not make sense to completely fill it with plastic. So a fill pattern provides the part's strength without using too much filament that may add weight to the part.\sidenote{The honeycomb conjecture states that the optimal way to divide a surface into regions of equal area with the least amount of perimeter is to have a regular hexagonal grid. The ratio of the perimeter to area is $\sqrt[4]{12}$ which has been proved mathematically to be a minimum for the honeycomb structure.} 

\textbf{Inill Density :}  As a designer, you have control over how much fill is printed by the fill density. The fill density is expressed as a percentage, so that 20\% means that one-fifth of the surface will be covered by plastic and the rest left empty. 

\textbf{Support :} Bridges are regions of a printed item that are not supported underneath by a previous layer of plastic when it is being printed. Without a supporting structure there is a limit to the length and angle to the horizontal that bridges can be successfully printed. Supports can be added by slicing programs that help to print bridge structures. 

\textbf{Print Speed :} The print speed directly affects the feasibility and quality of the prints. With fused filament fabrication systems, the mass of the molten filament leaving the nozzle is conserved so that a faster print speed means less material is deposited, and for good prints a thinner layer is needed for the same nozzle diameter. [SL systems similar effect in conservation of photon number...]

\subsection{G-code : The 3D Printer Machine Code}

The g-code contains the specific commands to the firmware that controls the stepper motors on the printer. There are two main types: \textbf{G***} commands move the printhead and the feedrate; \textbf{M***} commands set parameters of the system. See appendix 1 of this chapter for a reference list of g-code commands. 

\newpage
\section{Step 4: Machine Set-Up}

So this could include several different aspects of machine set-up which for FDM could be the temperature and levelling of the heated bed, setting the zero of the coordinate axis, the 

Machine set-up: Before printing, the machine will have to be set-up. This may include the levelling of the bed, homing of the printhead, or warming up heated extruder. This is usually handled by the slicing software. 

\newpage
\section{Step 5: Printing }

Print 

\newpage
\section{Step 6: Removal and Clean-up}

Part removal and clean-up : The part must be separated from the bed and cleaned up ready for the application. This could include removing any unwanted structures from the part, or chemical processing or sanding to smooth surfaces down. 

\subsection{Support and burr removal}

Printing overhanging structures may require support features and these will need to removed to finalise the model. The first step is cutting them away which can be achieved using fine pair of snips. Unfortunately this may leave a burr on the surface of the part which needs to be removed through sanding. 

\subsection{Smoothing a model's surface}

The layered printing process may result in a stepped appearance on the surface and it is desirable to smooth this out. There have been various attempts to do this: 
\begin{itemize}
 \item Sanding - using coarse to fine grit paper in circular motion in all stepped regions helps to smooth out the surface. This is applicable to the majority of printed materials. 
 \item Lacquer - XTC-3D has been used with PLA before. 
 \item Chemical vapour smoothing - Acetone is used with some plastics, notable ABS, but is not applicable to others (i.e. PLA). Other chemicals can be used for PLA, such as Ethyl Acetate. 
\end{itemize}

In order to obtain a smooth surface 

[Support removal] 

[Acetone smoothing]
\url{http://airwolf3d.com/2013/11/26/7-steps-shiny-finish-on-abs-parts-acetone/}
\url{http://3dprinting.stackexchange.com/questions/11/how-do-i-give-3d-printed-parts-in-pla-a-shiny-smooth-finish}

\url{http://blog.fictiv.com/posts/ultimate-guide-to-finishing-3d-printed-parts}
\url{http://makezine.com/projects/make-34/skill-builder-finishing-and-post-processing-your-3d-printed-objects/}
\url{https://www.youtube.com/watch?v=GSKxycs3kPg}

Taxonomy of Z-axis defects... 
\url{https://www.evernote.com/shard/s211/sh/701c36c4-ddd5-4669-a482-953d8924c71d/1ef992988295487c98c268dcdd2d687e}


\newpage
\begin{landscape}
\vspace*{-2cm}
\section{Appendix 1 : G-Code Commands}%
\begin{longtable}{c|c|c}
%\begin{tabular}{c|c|c}
\pagebreak \hline \hline  \textbf{Command} & \textbf{Parameters} & \textbf{Description} \\
\endfirsthead 
\hline \hline  \textbf{Command} & \textbf{Parameters} & \textbf{Description} \\
\hline
\endhead
\hline \multicolumn{3}{r}{\emph{Continued on next page}}
\endfoot
\endlastfoot

\hline  \multicolumn{3}{l}{\textbf{Power Control}} \\
  M80	& none	& Turn on ATX Power (if neccessary)\\
  M81 	& none & Turn off ATX Power (if neccessary)\\
  M40	& none & Eject part (if possible) \\
  \hline \multicolumn{3}{l}{\textbf{SD Card Filesystem Control}} \\
  M20 	& none	& List files at the root folder of the SD Card \\
  M21	& none  & Initialise (mount) SD Card \\
  M22	& none  & Release (unmount) SD Card \\
  M23	& Filename & Select File for Printing \\
  M24	& none 	& Start / Resume SD Card Print (see M23) \\
  M25	& none & Pause SD Card Print (see M24) \\
  
  
  M26	& Bytes & Set SD Position in bytes \\
  M27	& none	& Report SD Print status \\
  M28	& Filename & Write programm to SD Card \\ 
  M29	& Filename & Stop writing programm to SD Card \\
  \hline \multicolumn{3}{l}{\textbf{Extruder Control}} \\
  M101	& none & Set extruder 1 to forward (outdated) \\ 
  M102	& none & Set extruder 1 to reverse (outdated) \\
  M103	& none & Turn all extruders off (outdated) \\
  M104	& Temperature & Set extruder temperature (not waiting) \\
  M105 	& none & Get extruder Temperature \\
  M108	& none & Set extruder speed (outdated) \\ 
  M109	& Temperature & Set extruder Temperature (waits till reached) \\
  M113	& <PWM [S]> & Set Extruder PWM to S (or onboard potent. If not given) \\
  M126	& Time[P] & Open extruder valve (if available) and wait for P ms \\
  M127	& Time[P] & Close extruder valve (if available) and wait for P ms \\
  M128 & PWM[S] & Set internal extruder pressure S255 eq max \\
  M129	& Time[P] & Turn off extruder pressure and wait for P ms \\
  M143	& Degrees[S]	& Set maximum hot-end temperture \\
  M160 	& No.[S] 	& Set number of materials extruder can handle \\
  T	& No. & Select extruder no. (starts with 0) \\
  \hline \multicolumn{3}{l}{\textbf{Miscellaneous}} \\
  M41	& none & Loop Programm(Stop with reset button!) \\
  M42	& none & Stop if out of material (if supported) \\ 
  M43	& none & Like M42 but leave heated bed on (if supported) \\
  M84 	& none & Stop idle hold (DO NOT use while printing\!) \\
  M92	& Steps per unit & Programm set S steps per unit (resets) \\ 
  M106	& PWM Value & Set Fan Speed to S and start \\
  M107	& none & Turn Fan off \\ 
  M110	& Line Number & Set current line number (next line number = line no. +1) \\
  M111	& Debug Level[S] & Set Debug Level \\ 
  M112	& none & Emergency Stop (Stop immediately) \\
  M114 	& none & Get Current Position \\
  M115	& none & Get Firmeware Version and Capabilities \\
  M116	& none & Wait for ALL temperatures \\
  M117	& none & Get Zero Position in steps \\
  M119	& none & Get Endstop Status \\ 
  M140	& Degrees[S]	& Set heated bed temperature to S (not waiting) \\
  M141	& Degrees[S]	& Set chamber temperature to S (not waiting) \\
  M142 	& Pressure[S]	& Set holding pressure to S bar \\
  M203	& Offset[Z]	& Set Z offset (stays active even after power off) \\
  M226	& none	& Pauses printing (like pause button) \\
  M227	& Steps[P/S] & Enables Automatic Reverse and Prime \\
  M228	& none 	& Disables Automatic Reverse and Prime \\
  M229	& Rotations[P/S] & Enables Automatic Reverse and Prime \\
  M230	& [S] & Enable / Disable wait for temp.(1 = Disable 0 = Enable) \\ 
  M240	& none & Start conveyor belt motor \\
  M241	& none & Stop conveyor belt motor \\
  M245	& none & Start cooler fan \\
  M246	& none & Stop cooler fan \\
  M300	& Freq.[S] Duration[P] & Beep with S Hz for P ms \\
  \hline \hline 
%  \end{tabular}
\caption{A listing of g-code commands}
\end{longtable}
\end{landscape}


\chapter{Accessible Design : The OpenSCAD Language}
\label{ch:openscad}
\section{Introduction}

The OpenSCAD design language is a free programmatic design language that has taken hold within the hobbyist community. Like many other programming language, and OpenSCAD is one of the easiest to learn, the user writes a plain-text file which is interpreted by the OpenSCAD compiler to create the object. Once the designer is happy with it, they can export the object's surface as a STL file ready for printing. 

For the Physicist, it gives the opportunity to produce the necessary strcutrue using the few lines of codes; this is true especially for some of the interesting natural structures, such as Fractals, that one might wish to visualize. 

\begin{figure}[h]
 \centering
 \includegraphics[width=0.9\textwidth]{./graphics/openscad_screenshot.png}
 % openscad_screenshot.png: 0x0 pixel, 300dpi, 0.00x0.00 cm, bb=
 \caption{A screenshot of the OpenSCAD graphical user interface running under MacOS. The left-hand side of the screen gives the script file and the right-hand side gives the 3D visualization of the structure.}
\end{figure}

\subsection{Installation Details}

OpenSCAD is hosted at \url{http://www.openscad.org/} and all installation details are given on this website. Extra OpenSCAD information and help is available at \url{https://en.wikibooks.org/wiki/OpenSCAD_User_Manual}. 

So the roadmap for this chapter is as follows: First let's through the mathematical background that you need to know in order to be able to more complicated structures. Don't worry, OpenSCAD is particularly accessible and you can start to produce printable straight away but the mathematical background is  really needed to produce larger structures in a modular and parametrized manner. 

\subsection{Typical Usage}

Whilst the GUI and OpenSCAD is straightforward, it might help to go through the typical usage. This is along these lines: First you write your code describing your structure, probably rendering it a few times as you iterate the design\sidenote{The OpenSCAD GUI has two modes to render designs: using the OpenCSG (preview) or CGAL (render) engines. The preview rendering mode is operated by presssing F5, or go through the Design-Preview menu option, and the rendering mode by F6 or through the Design-Render menu option.}


\section{Geometric and Mathematical Background Summary}
\label{sn:mathematical}

\emph{In many design cases it is possible to use trial-and-error to place one part relative to another, for instance. However, trying to do this for many parts, when exploiting the programmatic nature of OpenSCAD, becomes tedious or impractical\sidenote{Alternatively in GUI-based CAD programs, they use the concept of constraints which, for example, can ensure two parts of the design touch each other at a particular point or that two surfaces lie perpendicular to each other.} So here background mathematical information is given in order for the reader to be able to write OpenSCAD code that expresses the required transformation}. 

This section is intended as a reminder of the mathematical background needed to be able to perform supporting calculations for AM design. It's not intended to replace a good textbook on vectors and mathematical geometry and the interested reader is referred to some of the textbooks in the further reading section of this chapter. 

We summarise the main points from mathematical geomtry below: 

\begin{itemize} 
\item Vectors are the mathematical representation of pointing to a location or a translation and are written as $\textbf{v} = \left( v_{1}, v_{2}, v_{3} \right)$, where the $v_{i}$ (i = 1,2,3) are called the components of the vector. In this instance, the vector has three components and represents a location in 3D space.
\item The distance to the 3D point is given by $d = \sqrt{ v_{1}^{2} + v_{2}^{2} + v_{3}^{2} }$, where $d$ is the distance to the 3D point
\item Unit vectors are special vectors whose distances are equal to 1, and typically represent pure directions. These are indicated by $\hat{\textbf{u}}$. 
\item The cosine of the angle between two unit vectors and another $\hat{\textbf{u}} = \left( u_{1}, u_{2}, u_{3} \right)$ is given by the dot product, where each component of the vectors is multiplied together:  $ \cos \theta_{uv} = \textbf{u}.\textbf{v} = u_{1}v_{1} + u_{2}v_{2} + u_{3}v_{3}$. 
\end{itemize}

Rotation and translation operations are given by the multiplication of a set of 3D points by a matrix. A rotation operation depends on the axis of rotation and angle $\theta$. For the perpendicular x,y, and z basis vectors directions, this is given by 
\begin{equation}
 \textbf{R}_{x} \left( \theta \right)= \begin{pmatrix} 1 & 0 & 0 \\ 0 & \cos \theta & \sin \theta \\ 0 & -\sin \theta & \cos\theta \end{pmatrix} 
\end{equation}
\begin{equation}
 \textbf{R}_{y} \left( \theta \right) = \begin{pmatrix} \cos \theta & 0 & \sin \theta \\ 0 & 1 & 0 \\ -\sin \theta & 0 & \cos\theta \end{pmatrix} 
\end{equation}
\begin{equation}
 \textbf{R}_{z} \left( \theta \right) = \begin{pmatrix} \cos \theta & \sin \theta & 0 \\  -\sin \theta & \cos\theta & 0 \\  0 & 0 & 1  \end{pmatrix} 
\end{equation}

These transformation operations can be applied one after another. For example, $\textbf{R}_{y} \left( 30^{\circ} \right)\textbf{R}_{z} \left( 20^{\circ} \right)$ corresponds to a rotation about the z-axis by 20$^{\circ}$ and then about the y-axis by 30$^{\circ}$. It's important to know that the order of the transformations matter. 

% The equation of a flat plane can be represented by a vector $\textbf{x}$ from the origin to any point on the plane, the plane's normal direction $\hat{\textbf{n}}$ and the distance from the origin to the plane (let's call this $d$). The equation of the plane is then $\hat{\textbf{n}}\cdot \textbf{x} = d$. 

\subsection{Transformations and OpenSCAD Recipes : Why is all this mathematics useful?}

There are a number of important recipes for the types of activities that are common to the Physicist/Designer: 

\textbf{1. Rotation about an arbitrary axis: } Really this is a generalisation of the above rotation matrices for any axis. When the axis is given by the unit vector $\hat{\textbf{u}}$, then the rotation is given by 
\begin{equation}
 R_{u} \left(\theta \right) = \cos \left( \theta \right) \textbf{I} + \sin \left( \theta \right) \left[ \hat{\textbf{u}} \right]_{\times}  + \hat{\textbf{u}} \hat{\textbf{u}}^{T}
 \label{arb_rot_matrix}
\end{equation}
where $\textbf{I} = \begin{pmatrix} 1 & 0 & 0 \\ 0 & 1 & 0 \\ 0 & 0 & 1 \end{pmatrix}$ is the unit matrix, $\left[ \hat{\textbf{u}} \right]_{\times} = \begin{pmatrix} 0 & -u_{z} & u_{y} \\ u_{z} & 0 & -u_{x} \\ -u_{y} & u_{x} & 0 \end{pmatrix}$, and $\hat{\textbf{u}} \hat{\textbf{u}}^{T} = \begin{pmatrix} u_{x}^{2} & u_{x}u_{y} & u_{x}u_{z} \\ u_{x}u_{y} & u_{y}^{2} & u_{y}u_{z} \\ u_{x}u_{z} & u_{y}u_{z} & u_{x}^{2} \end{pmatrix}$. 


Although this looks scary, it's quite simple to implement in the OpenSCAD language:
\begin{verbatim}
 
 rot_angle=60;// The rotation angle in degrees
 rot_axis = [1.0, 2.0, 3.0];// The rotation axis

 rotate(rot_angle, rot_axis)
 cylinder(r=10,h=1,$fn=60);// The object, a cylinder of radius 10 units and height 1 unit, with 60 angular divisions

 \end{verbatim}

\begin{figure}[h]
 \centering
 \includegraphics[width=0.4\textwidth]{./graphics/openscad_rotation_axis_null.png}%
 \hspace{0.1cm}\includegraphics[width=0.4\textwidth]{./graphics/openscad_rotation_axis.png}
 % openscad_rotation_axis.png: 0x0 pixel, 300dpi, 0.00x0.00 cm, bb=
 \caption{The effect of the rotation about the arbitrary axis before (left) and after (right) the effect of the operation.}
\end{figure}

An important part of the design process is fitting together two parts. 

\textbf{2. Finding the line joining two parts :} Ok so we are designing something that looks like a bizarre dumbell and we would like to join the two directions. The two flat ends of the dumbell are planes so have equations $\hat{\textbf{n}}_{1} \cdot \textbf{x} = d_{1}$, and $\hat{\textbf{n}}_{2} \cdot \textbf{x} = d_{2}$. The joining line is perpendicular to both normals $\hat{\textbf{n}}_{1}$ and $\hat{\textbf{n}}_{2}$, so it is parallel to their cross-product $\hat{\textbf{n}}_{1} \times \hat{\textbf{n}}_{2}$. 
The equation of the line between the two ends is
\begin{equation}
 \textbf{r} = c_{1} \hat{\textbf{n}}_{1} + c_{2} \hat{\textbf{n}}_{2} + \lambda \left( \hat{\textbf{n}}_{1} \times \hat{\textbf{n}}_{2} \right)
\end{equation}
where $c_{1} = \frac{d_{1} - d_{2} \hat{\textbf{n}}_{1} \cdot \hat{\textbf{n}}_{2} }{1 - \left( \hat{\textbf{n}}_{1} \cdot \hat{\textbf{n}}_{2} \right)^{2} }$ and $c_{2} = \frac{d_{2} - d_{1} \hat{\textbf{n}}_{1} \cdot \hat{\textbf{n}}_{2} }{1 - \left( \hat{\textbf{n}}_{1} \cdot \hat{\textbf{n}}_{2} \right)^{2} }$. 

So here let's say that we have a hex nut which we would like to mate on the outside of a plane. The nut can be succintly represented 
\begin{verbatim}
 
 difference(){
  cylinder(r=10,h=1,$fn=6,center=true);// The object, a cylinder of radius 10 units and height 1 unit, with 60 angular divisions
  cylinder(r=8,h=2,$fn=60,center=true);// The object, a cylinder of radius 10 units and height 1 unit, with 60 angular divisions
 }
 
 \end{verbatim}

And now we have our mating part, the 

\begin{verbatim}

 rotate(25,[1,1,1])// This rotates the part by 25 degrees around the (1,1,1) vector
 translate([0,0,30])// This translates the part by 30 units along the z-axis
 rotate([180,0,0])// This flips the part over 
 difference(){
     
      cylinder(r=10,h=5,$fn=6,center=true);// The object, a cylinder of radius 10 units and height 1 unit, with 6 angular divisions. The centre of the cylinder is centred on the origin.  

      cylinder(r=8,h=10,$fn=60);// The object, a cylinder of radius 10 units and height 1 unit, with 6 angular divisions. This cylinder base is centred on the origin.  

     
 } 
 \end{verbatim}

 Ok, in this example, the normal vector to the nut is along $\hat{\textbf{n}}_{1} = \left( 0, 0, 1 \right)$, which is the default normal direction for cylinders in OpenSCAD. 
 
 Before the operations are applied, the normal for the mating part is also $\hat{\textbf{n}}_{2} = \left( 0, 0, 1 \right)$. Then there is a rotation about the $x$-axis by 180 degrees, this changes the normal to $\hat{\textbf{n}}_{2} = \left( 0, 0, -1 \right)$. Then there is a translation, which leaves the normal unchanged. 
 
 Finally in the chain of operations, there is a rotation about the $\textbf{u} = (1,1,1)$ axis by 25 degrees. We can use the results from equation \ref{arb_rot_matrix} to calculate the rotation matrix of this transformation. To do this first we need to find the matrices $\left[ \hat{\textbf{u}} \right]_{\times}$ and $\hat{\textbf{u}} \hat{\textbf{u}}^{T}$. First we need to normalize the vector $\textbf{u}$ by its length, $\hat{\textbf{u}} = \tfrac{1}{\sqrt{3}} (1,1,1)$, then substitute this in to the matrices: 
 \begin{equation}
    \left[ \hat{\textbf{u}} \right]_{\times} = \frac{1}{\sqrt{3}}  \begin{pmatrix} 0 & -1 & 1 \\ 1 & 0 & -1 \\ -1 & 1 & 0 \end{pmatrix}
 \end{equation}
 and 
\begin{equation}
  \hat{\textbf{u}} \hat{\textbf{u}}^{T} = \frac{1}{3} \begin{pmatrix} 1 & 1 & 1 \\ 1 & 1 & 1 \\ 1 & 1 & 1 \end{pmatrix}. 
\end{equation}
Then putting this all together gives
\begin{equation}
 R_{u} \left(\theta \right) = \cos \left( \theta \right) \begin{pmatrix} 1 & 0 & 0 \\ 0 & 1 & 0 \\ 0 & 0 & 1 \end{pmatrix} + \sin \left( \theta \right) \frac{1}{\sqrt{3}}  \begin{pmatrix} 0 & -1 & 1 \\ 1 & 0 & -1 \\ -1 & 1 & 0 \end{pmatrix}  + \frac{1}{3}\begin{pmatrix} 1 & 1 & 1 \\ 1 & 1 & 1 \\ 1 & 1 & 1 \end{pmatrix} 
\end{equation}


 

 
 Move a part to mate with another


\section{Difference and Union Operations}

One application of 3D printing to the Physics is the production of feedthrough panels, such as from the laboratory to the interior of a light-tight box or when printing an enclosure. Feedthrough panels allow cables, for instance, to pass from the lab bench through into the enclosure where the signal is required. When this is required, you need to know what size of hole. 

\begin{verbatim}
 difference(){
    
 
 }
\end{verbatim}






\section{OpenSCAD Design Exercises}
\label{sn:design_exercises}
\subsection{Basic CSG Shapes}

For these exercises, and firstly, make sure you are able to see the axes in the render window. 

\begin{enumerate}
 \item \textbf{Basic CSG shapes}: Create the following
  \begin{enumerate}
    \item A sphere with a radius of 10 units and 10 angular divisions. 
    \item A sphere with a radius of 30 units and angular divisions every 3 degrees
    \item A cylinder with radius of 10 units, height of 20 units and angular divisions every 6 degrees
    \item Using a cylinder command, a hexagonal-based prism which is 20 units long. 
    \item A cone with a base diameter of 20 units and top diameter of 1 unit. 
    \item A unit cube 
    \item A square-based prism which is 4 units long and 1 by 1 unit in cross-section, oriented with the long axis parallel to the Z-axis. 
    \item A L-shape with a cross-section of 2 units by 2 units, in which one arm is 5 units long and the other is 8 units long
  \end{enumerate}
 \item \textbf{Polyhedra}: Create the following
  \begin{enumerate}
  \item Enumerate, starting from 0, and write down the vertices of a square-based pyramid which is 10 units by 10 units, and 20 units high 
  \item For each face of the square based pyramid, write down the triplet of numbers of each face. 
  \item Using a polyhedra command, and information from the previous two questions, render the square-based pyramid using OpenSCAD. 
  \item Increase the height of the pyramid to 40 units and re-render
  \end{enumerate}
\end{enumerate}

\newpage
\subsection{OpenSCAD Design Exercises : Transformations}

\begin{enumerate}
 \item \textbf{The 'center' option\sidenote{OpenSCAD uses American-English spellings.}: } For the following OpenSCAD code\sidenote{Note that the transformation modifying statement (for example the color or rotation commands) do not have a semi-colon at the end of the line. This indicates to the OpenSCAD engine that these operations apply to the following CSG object (which does end with a semi-colon)}.
 \begin{enumerate}
 \item[ ]
 \begin{quote}
  \begin{verbatim}
    color("blue") 
    cylinder(r=1,h=5,$fn=20, center=true);  
    
    color("red") 
    cylinder(r=1,h=5,$fn=20);// same as center=false
    
    color("cyan") 
    cube([2,3,4], center=true);
    
    color("magenta") 
    cube([2,3,4]);
  \end{verbatim}
 \end{quote}
  \item Try commenting one of these lines out using \/\/ at the start of the line 
  \item Calculate the translation vector between the bases of the red and blue cylinders, and then between the cyan and magenta cubes 
  \item Write an equation for the cylinder's centering translation in terms of the cylinder radius, r, and its height, h. Repeat this in terms of a cuboid's x,y, and z sides of lengths a,b, and c respectively. 
 \end{enumerate}

 \item \textbf{Translations :}\sidenote{For these exercises, make sure you keep the axes visible using the View-Show Axes menu options.}
 \begin{enumerate}
 \item[ ]
 \begin{quote}
  \begin{verbatim}
    translate([2,0,1]) 
    color("red") 
    sphere(r=2,$fn=40);
    
    translate([0,7,7]) 
    color("green") 
    sphere(r=3,$fn=40);
    
    translate([1,0,13]) 
    color("blue") 
    sphere(r=1,$fn=40);
  \end{verbatim}
 \end{quote}
 \item Move the spheres so that they are just touching each other in a line, and record the translation vectors 
 \item Decrease the fn value to 7 for all three spheres and note what happens. Why might this be a problem? 
 \end{enumerate}
 
 \item \textbf{Rotations :} 
 \begin{enumerate}
 \item[ ]
 \begin{quote}
  \begin{verbatim}
    rotate([0,0,0]) 
    translate([sqrt(2)*30,0,0]) 
    color("red") 
    sphere(r=3,$fn=40);
    
    translate([0,30,30]) 
    color("green") 
    sphere(r=3,$fn=40);
 \end{verbatim}
 \end{quote}
 \item Set and record the rotation angles so that the red sphere overlaps the green sphere
 \end{enumerate}
 
 \item \textbf{Scaling :} 
 \begin{enumerate}
 \item[ ]
 \begin{quote}
  \begin{verbatim}
    scale([20,1,3])
    color("red")
    sphere(r=3,$fn=40);
  \end{verbatim}
 \end{quote}
  \item Find the OpenSCAD GUI icon which fits the object in the window
  \item Describe the effect that this scale command has had on the sphere
\end{enumerate}
 
 \item \textbf{Mirror Transformation:} 
 \begin{enumerate}
 \item[ ]
 \begin{quote}
 \begin{verbatim}
    mirror([0,1,0])
    color("red")
    translate([2,0,0])
    cube([4,1,1]);
    
    color("green")
    translate([2,0,0])
    cube([4,1,1]);
    
    rotate([0,0,45]) 
    color([1.0,1,0.4,0.6]) 
    cube([0.01,10,10],center=true);
 \end{verbatim}
 \end{quote}
 \item Note that the color command accepts the vector [r,g,b,a], where [r,g,b] are the red,green,blue values respectively (from 0 to 1) and a is the transparency (from 0 to 1, where 0 is transparent and 1 is opaque). 
 \item Set the mirror comand vector in the first example so that the green cuboid is correctly mirrored by the red cuboid in the yellow mirror
 \item[ ]
 \begin{quote}
 \begin{verbatim}
    color("red") 
    translate([5,5,5]) 
    cube([1,1,1],center=true);
    
    mirror([1,0,0]) 
    color("green") 
    translate([5,5,5]) 
    cube([1,1,1],center=true);
    
    mirror([0,1,0]) 
    color("blue") 
    translate([5,5,5]) 
    cube([1,1,1],center=true);
    
    mirror([0,0,1]) 
    color("magenta") 
    translate([5,5,5]) 
    cube([1,1,1],center=true);
 \end{verbatim}
 \end{quote}
 \item In the second example, add mirror commands for the given cubes to give one cube in every octant
 \end{enumerate}

 \item \textbf{Arbitrary Transformations:} Arbitrary transformations are also possible by using the multmatrix, which multiplies all points on the surface of the object by a 4x4 matrix. It is a convenient way of representing translations as well as other transformations, by adding a fourth element with the value 1 to the 3D point to be transformed. For example, the following 4x4 matrix is simply a translation:
 \begin{eqnarray}
  \begin{pmatrix} x_{2} \\ y_{2} \\ z_{2} \\ 1 \end{pmatrix} & = &  \begin{pmatrix} 1 & 0 & 0 & t_{1} \\ 0 & 1 & 0 & t_{2} \\ 0 & 0 & 1 & t_{3} \\0 & 0 & 0 & 0 \end{pmatrix} \begin{pmatrix} x_{1} \\ y_{1} \\ z_{1} \\ 1 \end{pmatrix} \\
  & = &  \begin{pmatrix} x_{1} + t_{1} \\ y_{1} + t_{2} \\ z_{1} + t_{3} \\ 1 \end{pmatrix} 
\end{eqnarray}

\begin{enumerate}
 \item This exercise concerns equivalent transformations and also shows the use of variables in OpenSCAD scripts 
 \begin{quote}
 \begin{verbatim}
	angle=20;// degrees
	M = [ [A, B, 0, 0],[C, D, 0, 0],[0, 0, 1, 0]];
	
	color("red") 
	multmatrix(M) 
	cube(10,center=true);
	
	rotate([0,0,angle]) 
	color("green") 
	cube([5, 5, 20],center=true);
 \end{verbatim}
\end{quote}
\item Calculate and replace A,B,C, and D with sin and cos functions that correctly transform the red cube so that it's sides are parallel with the green cube. Comment on the top left 3x3 matrix within the matrix M. 
 \begin{quote}
 \begin{verbatim}
    M1 = [	
		[ 1, 0.6, 0.0, 20 ], 
		[ 0, 1,   0.7, 0  ],
		[ 0, 0,   1,   0  ],
		[ 0, 0,   0,   1  ] 
	 ] ;
    
    color("red") 
    multmatrix(M1) 
    cube(10,center=true);
    
    M2 = [ 	
		[ 1, 0, 0, 20 ], 
		[ 0, 1, 0, 0  ],  
		[ 0, 0, 1, 0  ],
		[ 0, 0, 0, 1  ] 
	 ] ;
    color("red") 
    multmatrix(M2) 
    cube(10,center=true);
 \end{verbatim}
\end{quote}
\item Describe the two matrix operations M1 and M2 and how they transform the cube
\item Create a cone with base radius 10 units and rotate this object by 60 degrees about the axis $\left(1,1,1\right)$.
 \end{enumerate}


 
 \end{enumerate}% End of transformations section 
 
\newpage
\subsection{OpenSCAD Design Exercises : Special Transformations}


\begin{enumerate}
 \item \textbf{Minkowski Sum: } The Minkowski sum is the object equivalent to the convolution of one object with another, and is a good way to produce curved edges on parts. It is computationally expensive so if you'd like to use this operation then expect to wait a short while on less capable computers. 
 \begin{enumerate}
 \item[ ]
 \begin{quote}
 \begin{verbatim}
 
 translate([-20,0,0])
 cube([10,10,1],center=true); // Object 1
 
 translate([0,0,0])
 sphere(3,$fn=6,center=true); // Object 2
 
 // This translation applies to the whole combined object
 translate([20,0,0])
 minkowski(){
	sphere(3,$fn=6,center=true); // Object 2
	cube([10,10,1],center=true); // Object 1
}

 \end{verbatim}
\end{quote}
\item Using a Minkowski sum operation, create a box and lid whose edges are nicely curved. 
\end{enumerate}
\item \textbf{Convex Hull: }
\begin{enumerate}
 \item[ ]
 \begin{quote}
 \begin{verbatim}
translate([-40,0,0])// This translation applies to object 1
color("blue") 
sphere(10,$fn=32,center=true); // Object 1 on its own

translate([-20,0,0])// This translation applies to object 2
color("red") 
cylinder(r=5,h=10,$fn=16,center=true); // Object 2 on its own

translate([0,0,10])
color("blue") 
sphere(10,$fn=32,center=true); // Object 1 translate along Z by 10 units

color("red") 
cylinder(r=5,h=10,$fn=16,center=true); // Object 2 at the origin

// This translation applies to the whole combined object
color("magenta") 
translate([20,0,0])
hull(){
	translate([0,0,10])
	sphere(10,$fn=32,center=true); // Object 1
	cylinder(r=5,h=10,$fn=16,center=true); // Object 2 on its own
	cube([10,10,1],center=true); // Object 1
}
 \end{verbatim}
\end{quote}
\item Describe what is meant by convexity of a set of points 
\item Describe a feature of a design where and why you might want to use this operation. 
 \end{enumerate}
\end{enumerate}





\newpage
\subsection{OpenSCAD Design Exercises : To practise and for fun! }

\begin{itemize}
 \item Write your name's initials using only cylinders, translate and rotate commands 
 \item Design an enclosure for an Arduino Uno (you'll need the board dimensions which are here: \url{https://www.flickr.com/photos/johngineer/5484250200/sizes/o/in/photostream/}) 
 \item Create a flat 8x8 checkerboard of colored squares, and then pieces in a stylised gometric manner (have look at this \url{http://www.thingiverse.com/search/page:1?q=chess+pieces&sa=} for inspiration!)
\end{itemize}

\newpage
\section{Further Reading}

\begin{itemize}
 \item Functional Design for 3D Printing : Designing 3D printed things for everyday use, by Clifford T Smyth. 
\end{itemize}


\section{Appendix 1 : OpenSCAD CheatSheet}%

\begin{fullwidth}
  \parbox[c][\textwidth][s]{\textheight}{%
  \includegraphics[width=2\textwidth, keepaspectratio, trim=10 250 10 10,clip=true]{./graphics/OpenSCAD_CheatSheet.pdf}
  \captionof{figure}{A reproduction of the OpenSCAD cheat sheet, giving details of the majority of commands used.}
} 
\end{fullwidth}




\chapter{Stereolithography}
\label{ch:fff}
\section{\SL Process Description}

[Picture of the \SL system]

\SL was developed in the mid-80s by **** working for the ****. 

The components of a \SL system comprise a vat of photo-sensitive gel, a laser and a motorised platform. \SL prints objects by layers and, for each layer of the printed item, the laser moves to each point in that layer which is required to be solid. The laser initiates a polymerisation processes which results in the gel's solidfication near the laser. 


\section{Laser Properties}


Within a \SL system the laser initiates the curing process and is specified by its power and wavelength. For the purposes of modelling the \SL process we need to find out how to calculate the rate of photons hitting the upper surface of the resin from these specifications. 

\subsection{How much energy is carried by a single photon?} 

Here we calculate how much energy is transmitted by a single photon from the \SL laser system. 

Formlabs are a leading company that produce \SL printers so as an example let's use their specification of the Formlabs One \SL printer. The Formlabs Two \SL printer has a 405nm Violet laser with 250 mW maximum power \sidenote{For the specification of the Formlabs two, please see: http://formlabs.com/products/3d-printers/tech-specs/}. 

So, in order to calculate the energy carried by a single photon of wavelength $\lambda$, we first use the relationship
\begin{equation}
E = \frac{hc}{\lambda}, 
\end{equation}
where $h$ is Planck's constant and $c$ is the speed of light. So putting in the numbers gives $E = \frac{6.63 \times 10^{-34} \times 3 \times 10^{8} }{6.63 \times 10^{-9}} = 4.9 \times 10^{-19}$ J. 

It is sometimes conventional to use the units of electron-volts\sidenote{The electron volt is defined as the energy given to an electron by an accelerating potential of 1V and is given the synbole eV.} rather than the SI unit of joules (whose symbol is J) when talking about energies. In order to convert between them we divide by the charge of an electron, or $1.6 \times 10^{-19}$ C. The energy of a single photon from a 405nm laser is then $4.9 \times 10^{-19} / 1.6 \times 10^{-19} = 3.1$eV. 

Why does this matter? The solidification process is initiated by the laser photons and requires the breaking of bonds within the gel's polymers. These bonds cannot be broken by photons of lower energies, or longer wavelengths, such as those in the infra-red (500nm and upwards). If we were to try to use an infra-red laser then we'd find that the gel is transparent for this wavelength. This particular property is exploited in the two-photon polymerisation printing technique which is described later in section ***. 

\subsection{How many photons are delivered per second by a typical Stereolithography laser?} 

The speed at which the initiation occurs over the laser's spot region is governed by the rate photons hiting this region. So we're also able to estimate this: Returning to the original Formlabs two specification as our illustrative example, we would like to calculate the laser intensity from its power specification of 250 mW. The maximum laser power is 250mW\sidenote{Watts are the derived SI unit for power, which is given in joules per second.} or $250 \times 10^{-3}$ J/s; this is the maximum amount of energy transferred by the photons in the laser's beam. As we've already found that the energy per photon is $4.9 \times 10^{-19}$~J so the rate of photons is $250 \times 10^{-3} / 4.9 \times 10^{-19} = 5.1 \times 10^{17}$~photons per sec. 

[Summary of calculation methods?]

\section{Laser Intensity and its Attenuation in Resin}

The photons in the laser beam enter the resin perpendicularly at its surface and travel through the resin until they interact with a polymer and start the solidification process. As the photons travel further into the gel and it has passed by more resin, it is increasingly likely that more and more become absorbed by the resin. We model this by expressing the fractional loss of the intensity \sidenote{We define the photon beam intensity as the number of photons per unit area of the gel's surface, per unit time.} of photons is proportional to the distance travelled through the resin. This is also called the Beer-Lambert law and can be expressed as 
\begin{equation}
\frac{- dN}{N} \propto dz
\end{equation}
where a small number $dN$ of photons have been lost (hence the negative sign) in a small distance $dz$; we take the $z$-axis as the direction downwards into the resin starting at $z=0$ at the resin's surface. 

We then set the proportonality constant to be $1 / D_{p}$, where $D_{p}$ is called the penetration depth, and so we can then write 
\begin{equation}
\frac{- dN}{N} = \frac{dz}{D_{p}}. 
\end{equation}

At the resin's surface, $z=0$, the laser intensity is $N_{0}$ (this is a constant, and we will find out later how to relate this to the laser power), The laser intensity decreases to $N\left(z\right)$ at a distance $z$ into the resin (which we would like to find out how this varies). 

Integrating both sides gives 
\begin{equation}
\int_{N_{0}}^{N\left(z\right)} \frac{- dN}{N} = \int_{0}^{z} \frac{dz}{D_{p}}. 
\label{N0}
\end{equation}
The left-hand side of equation \ref{N0} is 
\begin{eqnarray}
\int_{N_{0}}^{N\left(z\right)} \frac{- dN}{N} & = & \ln \left[ N\left(z\right)\right]  - \ln \left[ N_{0} \right]. \\
& = & \ln \left[ \frac{N\left(z\right)}{ N_{0} } \right]. \\
\end{eqnarray}
The right-hand side of equation \ref{N0} is just $\frac{z}{D_{p}}$. 

Equating the left and right hand sides gives 
\begin{equation}
 - \ln \left[ \frac{N\left(z\right)}{ N_{0} } \right] = \frac{z}{D_{p}}. 
\end{equation}
The next steps then take the negative sign over to the right hand side of this equation and take the exponential of both sides. This then results in 
\begin{equation}
\frac{N\left(z\right)}{ N_{0} } = \exp \left\{ - \frac{z}{D_{p}}. \right\}
\end{equation}
and then finally the main result: 
\begin{equation}
N\left(z\right)  = N_{0} \exp \left\{ - \frac{z}{D_{p}}. \right\}
\end{equation}

The physical significance of the penetration depth is that it is the distance after which the laser beam has lost $\exp \left\{ -1 \right\}$, or 37\%, of its intensity. The penetration depth is a property of the choice of resin and expresses the degree to which the resin attenuates the laser beam. 

For example, at a distance of 1mm into the resin, the intensity falls to a fraction of $\frac{N\left( z = \textrm{1 mm} \right)}{ N_{0} } =  \exp \left\{ - \frac{1}{D_{p}} \right\}$, where $D_{p}$ is also measured in mm. So for a highly-attenuating resin with a penetration depth of 0.1mm, we find the beam has fallen to $\exp \left\{ - \frac{1}{0.1} \right\} = 0.000045$ of its intensity at the surface of the resin. For a poorly attenuating resin, let's say that $D_{p} = 10$mm, the intensity, at 1mm into the resin, has fallen by $\exp \left\{ - \frac{1}{10} \right\} = 0.90$, or 90\% of its incident intensity at the surface.\\[1cm] 


\section{The solidfication region}

A common assumption to make is that the laser spot at the gel surface ($z=0$) has a Gaussian spread, so the intensity perpendicular to the beam is given by
\begin{equation}
 N\left(x,y,0 \right) = N\left(x=0,y=0,0 \right) \exp\left( - 2\frac{x^{2}+y^{2}}{\sigma_{r}} \right) 
\end{equation}
where the parameter $\sigma_{r}$ determines the spread of the beam. 

The Gaussian laser beam has cylindrical symmetry - it only depends on $r^{2} = x^{2}+y^{2}$ and it is often more convenient to write this as 
\begin{equation}
 N\left(r, 0 \right) = N\left(r=0, 0 \right) \exp\left( - \frac{r^{2}}{\sigma_{r}^{2}} \right) 
\end{equation}

The photon intensity over all spatial coordinates is then 
\begin{equation}
N\left(r,z\right)  = N\left(r=0,z=0\right) \exp \left\{ - \frac{z}{D_{p}} - \frac{r^{2}}{\sigma_{r}^{2}}  \right\}
\end{equation}

\subsection{Total laser power}

The total laser power\sidenote{We define the laser power here as the number of photons per unit time incident on the whole gel surface.} is given by integrating over the directions perpendicular to the laser's direction. Using cylindrical coordinates, the power integral is 
\begin{eqnarray}
 P & = & N\left(r=0, 0 \right) \int_{0}^{2\pi} \int_{0}^{\infty} N\left(r, 0 \right) r dr d\theta \\
   & = & N\left(r=0, 0 \right) \int_{0}^{2\pi} \int_{0}^{\infty} \exp\left( - \frac{r^{2}}{\sigma_{r}^{2}} \right) r dr d\theta \\
   & = & N\left(r=0, 0 \right) 2\pi \int_{0}^{\infty} \exp\left( - \frac{r^{2}}{\sigma_{r}^{2}} \right) r dr \\
   & = & N\left(r=0, 0 \right) \pi \sigma_{r}^{2} \int_{0}^{\infty} \exp\left( - u  \right) du \\
   & = & N\left(r=0, 0 \right) \pi \sigma_{r}^{2}
\end{eqnarray}
where $\theta$ is the angular coordinate, and the integral's variables were transformed by $u =  \frac{r^{2}}{\sigma_{r}}$. We can express the intensity of photons at the centre of the beam in terms of its power by a rearrangement of this
\begin{equation}
   N\left(r=0, 0 \right)  = \frac{P}{\pi \sigma_{r}^{2}}
\end{equation}

[Think we are missing the energy per photon as a factor in front of the expression]

\subsection{Total laser exposure and the shape of the solidified region }

The laser exposure is the total number of photons per unit area over a period of time\sidenote{The exposure is defined as the total number of photons that have been incident on the gel surface per unit area}. 

It's also important to consider the laser moving across the surface of the gel in a radial direction at a speed $v_{r}$
\begin{equation}
 E\left(x,y,z\right) = \frac{P}{\pi \sigma_{r}^{2}}  \int_{0}^{t} \exp \left\{ - \frac{z}{D_{p}} - \frac{x^{2}+y^{2}}{\sigma_{r}^{2}}  \right\} dt 
\end{equation}

Changing coordinates from the time to radial coordinate in the integral to 
\begin{equation}
 v_{x} = \frac{dx}{dt}
\end{equation}
and then 
\begin{equation}
 E\left(x,y,z\right) = \frac{Pv_{r}}{\pi \sigma_{r}^{2}}  \exp \left\{ - \frac{z}{D_{p}}\right\} \int_{0}^{v_{x} t}  \exp \left\{ - \frac{x^{2}+y^{2}}{\sigma_{r}^{2}}  \right\} dx 
\end{equation}


\begin{equation}
 E\left(y,z\right) = \frac{P v_{r}}{\pi \sigma_{r}^{2}}  \exp \left\{ - \frac{z}{D_{p}}\right\} \exp \left\{ - \frac{y^{2}}{\sigma_{r}^{2}}  \right\} \int_{0}^{v_{x} t}  \exp \left\{ - \frac{x^{2}}{\sigma_{r}^{2}}  \right\} dx
\end{equation}

To tidy up this expression, we write 
\begin{equation}
 E\left(y,z\right) = E_{max}  \exp \left\{ - \frac{z}{D_{p}}\right\} \exp \left\{ - \frac{y^{2}}{\sigma_{r}^{2}}  \right\} 
\end{equation}
where 
\begin{equation}
 E_{max} = \frac{P v_{r}}{\pi \sigma_{r}^{2}} \int_{0}^{v_{x} t}  \exp \left\{ - \frac{x^{2}}{\sigma_{r}^{2}}  \right\} dx. 
 \label{emax}
\end{equation}

As the integral does not depend on the $y$- or $z-$ coordinates, the shape of the solidified region is given by the factor in front of the integral. 
\begin{equation}
 E\left(y,z\right)  = E_{max} \exp \left\{ - \frac{z}{D_{p}}\right\} \exp \left\{ - \frac{y^{2}}{\sigma_{r}^{2}}  \right\} 
\end{equation}

The cured, or solid, region is where the exposure $E\left(y,z\right)$ is greater than a critical exposure $E_{c}$. The shape of the cured region is 
\begin{equation}
 E\left(y,z\right)  = E_{c} = E_{max} \exp \left\{ - \frac{z}{D_{p}}\right\} \exp \left\{ - \frac{y^{2}}{\sigma_{r}^{2}}  \right\} 
\end{equation}
or rearranging 
\begin{equation}
\ln \left( \frac{ E_{max}}{E_{c} } \right) =  \frac{z}{D_{p}} + \frac{y^{2}}{\sigma_{r}^{2}} 
\label{shape}
\end{equation}
which is a parabolic cylinder facing down from the gel's surface. 

\subsection{The cure depth}

The cure depth $C_{d}$ is the depth to which the solidified region extends and, from equation \ref{shape}, this maximum is at $y=0$. 
\begin{equation}
\ln \left( \frac{ E_{max}}{E_{c} } \right) =  \frac{C_{d}}{D_{p}} 
\label{cd}
\end{equation}
It is possible to control the layer height by selecting a gel resin with the right $E_{c}$ and $D_{p}$, control the laser power, which is related to $E_{max}$ through equation \ref{emax}. 

This allows a measurement of the penetration depth by varying the maximum exposure $E_{max}$, through the laser power and time, and plotting $\ln \left( E_{max} \right)$ against the measured cure depth; from equation \ref{cd} the gradient of this plot is $D_{p}$ and the intercept gives $E_{c}$. 

[example plot]

[worked example?]

\subsection{The linewidth}

The resolution of Stereolithographic prints can also be found from equation \ref{shape}. As the printer is usually concerned with the smallest feature that is possible, the finest resolution is limited by the widest point of the cured region. The widest point is at the surface, when $z=0$, and is 
\begin{eqnarray}
L_{w} & = & 2 \sigma_{r} \sqrt{ \ln \left( \frac{ E_{max}}{E_{c} } \right) } \\
L_{w} & = & 2 \sigma_{r} \sqrt{ \frac{C_{d}}{D_{p}} } \\
\label{lw}
\end{eqnarray}
So if one wanted to produce very small feature sizes then the printer would choose a small laser spot, and a material with a high penetration depth and small cure depth. 










\chapter{FFF types}
\label{ch:fff}
% \documentclass[]{article}
% % \usepackage[numbers,sort]{natbib}
% \usepackage{url}
% \usepackage{graphicx}
% \usepackage{amsmath}
% 
% 
% %opening
% \title{Fused Filament Fabrication Systems}
% \author{Chris Steer}
% 
% \begin{document}
% \bibliographystyle{unsrt}
% 
% \maketitle
% 
% \begin{abstract}
% This note describes two types of FF systems: Cartesian and delta 3D fused filament fabrication systems. 
% \end{abstract}

\section{Fused Filament Fabrication (FFF) Printing Systems}

Fused filament fabrication systems produce three-dimensional objects through the adding molten material to each layer at the designed positions. FFF system include: the motion system, which is a mechanism to move the hot end to the desired place in the print; the hot end and extrusion of the filament itself. 

\begin{figure}
\centering
\includegraphics[width=0.7\linewidth]{pictures/extrusion}
\caption{A schematic of fused filament deposition in which solid filament is pushed into a hot nozzle. Inside the hot nozzle, the filament liquefies and is pushed out of a small hole in order to form the material for the new layer. The nozzle and/or the build plate may move when placing new material on the printed part.}
\label{fig:extrusion}
\end{figure}

Commercial filament is standardised to 1.75mm and 2.85mm (why?). 

\section{Hot End Motion}

As they are inexpensive and easily controlled, stepper motors are commonly used in FFF printers. 

\subsection{Stepper Motors Description}

Control electronics are detailed in a later section. 

[NEMA standard motors - what does the designation mean]

[Discussion of the torque ratings and force required for motion]

\subsection{Printer Coordinate Systems}

Prior to starting the printing will need to home itself. When this is started the operator will see the printer move the carriages to their maximum extent touching microswitches. When all of the microswitches are pressed then the control system will consider the printer to be homed. This home posiiton sets the printer's origin of their coordinate system. 

\subsection{Cartesian Printers}

Cartesian printers use stepper motors to move the nozzle head in perpendicular directions. This system produces linear movement in one direction by stepping a motor attached to a drive mechanism corresponding to that direction; so there is a X-direction stepper motor, a Y-direction stepper motor, and a Z-direction stepper motor. 

There are variations in Cartesian designs that could be any of the following: 
\begin{itemize}
	\item As is the case with the Prusa design, there is may be more than one motor per axis. 
	\item The stepper motor drive mechanism can be a toothed belt, screw thread, or fishing line.
	\item The framework can vary significantly such as a bar (e.g. Reprap Ormerod), triangular (e.g. Reprap Mendel) to rectangular frame (e.g. Prusa variants); 
	\item The extruder design can also vary between printer types
	\item In some printer designs, the bed may move along one or more axes too. 
\end{itemize}
    
\begin{figure}
\centering
\includegraphics[width=0.7\linewidth]{pictures/ormerod-complete}
\caption{A picture of an assembled Reprap Ormerod. Note the bar design of the horizontal, typically labelled X, axis.}
\label{fig:ormerod-complete}
\end{figure}

\begin{figure}
\centering
\includegraphics[width=0.7\linewidth]{pictures/500px-Reprappro-Mendel}
\caption{A picture of an assembled Reprap Mendel with a characteristic triangular framework.}
\label{fig:500px-Reprappro-Mendel}
\end{figure}

\begin{figure}
\centering
\includegraphics[width=0.7\linewidth]{pictures/prusai3}
\caption{A picture of an assembled Prusa i3 printer with a vertical rectangular framework.}
\label{fig:prusai3}
\end{figure}

\subsection{Delta Printers}

The delta motion system was originally developed for PCB production in which electronic components are picked from their storage box and placed on the PCB. An example of this system is shown in Figure \ref{fig:pick_and_place}.  

\begin{figure}
\centering
\includegraphics[width=0.7\linewidth]{pictures/pick_and_place}
\caption{Example pick and place robots based upon the delta mechanism. }
\label{fig:pick_and_place}
\end{figure}

The Rostock printer is an archetypal delta printer. The three arms are each moved vertically up and down by stepper motors in the base of the printer. As the arms are fixed in length, and hinged at both joints on the carriage and effector, the head can be made to move to the desired position by changing the vertical height of one of the stepper motors. In order to maintain print speed, the head is made to be as light as possible with the extruder stepper motor separate from the nozzle. In Figure \ref{fig:rostock}, for example, there is a white ('Bowden') tube guiding the filament from the extruder stepper motor, which is mounted on the frame in the rear, to the nozzle. 

The benefits of the delta design over the Cartesian design are that it is typically faster and cheaper than Cartesian printers. Another marginal issue is that there are signficant computational demands of converting g-code to delta printer stepper motor movements. Consequently, the embedded computer control system is typically more powerful (and more expensive) than Cartesian printers. For some filaments which have a significant roughness on the reel, the Bowden tube can cause significant friction and create problems to do with the retraction of filaments in this design. There is a nice discussion of the benefits and disadvantages of delta printers at \cite{delta_url}. 

\begin{figure}
\centering
\includegraphics[width=0.7\linewidth]{pictures/450px-Rostock}
\caption{The Rostock 3D printer with its characteristic three vertical carriages; the carriages move up and down changing the position of the nozzle within the effector head just above printed voronoi Klein bottle. }
\label{fig:rostock}
\end{figure}


\subsection{Maths of Delta Coordinates}

The carriage supports of a Delta printer lie at the corners of a triangle. 


\subsection{Comparison of drive method}

Compare belt, string, or lead screw driven motion system


\section{Filament Motion}

The filament undergoes heating through its glass and then its melting transition. At the glass transition, the filament material becomes rubbery and expands increasing the pressure within the nozzle. The filament also grips the interior of the barrel and, if the melting zone is too long, may cause jamming in the nozzle. At the melting point the filament becomes liquid and so it can be extruded out from the small hole at the farthest point of the hot end\sidenote{For more information on the design aspects of different extruder designs can be found in references \cite{extruders, reprap_hotend}.}. 

\begin{figure}
\centering
\includegraphics[width=0.7\linewidth]{pictures/nozzle_schematic}
\caption{A schematic of a typical extrusion nozzle, indicating the melt zone area.}
\label{fig:nozzle_schematic}
\end{figure}

[Discussion of the torque ratings and force required for pushing the filament through]

The filament is pulled from its reel and into the nozzle. In some systems, mostly delta printers, an extra tube is used to guide the filament into the hot end\sidenote{This is known as a Bowden tube after ?????...}

\subsection{Filament Materials}

\subsection{Melt Zone}

For constant print speed, the rate of liquid filament leaving the extrusion nozzle and solid filament entering the top of the nozzle are also constant. The filament leaving the barrel reduces the presure in the melt zone which is balanced by the increase in pressure of new filament being pushed in from the top. Modelling the extruder as a laminar flow in a tube, the volume flow rate depends on the nozzle extrusion hole radius $r$, material's viscosity $\eta$, pressure drop in the tube $\Delta P$, the melt zone length $L$, and is given by 
\begin{equation}
\frac{dV}{dt} = \frac{\pi r^{4} \Delta P}{8 \eta L}. 
\end{equation}

For a given print run, the pressure in the melt zone can be altered by changing the filament feed rate; so more solid filament means a higher pressure inside the melt zone and a larger pressure difference between this and the outside. 

The size of the melt zone inside the nozzle is important, especially as to how responsive the nozzle is to fine details in contrast to printing at speed. Short melt zones give less time lag when the flow rate changed or filament retraction, especially when compared to longer melt zones. In contrast, longer melt zones permit faster printing through thicker layers, as a larger reservoir is available.  


%The nozzle's orifice lengths can range from 0.24mm to 1.32mm depending on nozzle model \cite{nozzle_orifices}. 

If the pressure in the barrel becomes too great then the extruder stepper motor may not be able to provide enough torque to push new filament into the melt zone. Consequently, the volume flow rate of filament is primarily limited by a limit in the maximum force applied to the new, solid filament and applied stepper motor torque. 

\subsection{Euler Buckling}

One failure mechanism that may occur is buckling of the filament, especially when the force applied to the solid ingoing filament is high. Euler derived a formula for the critical force $F_{\textrm{C}}$, at the point of buckling of a column fixed at both ends, which is given by 
\begin{equation}
F_{\textrm{C}} = \frac{4 \pi^{2} E I}{L_{\textrm{f}}^{2}}, 
\end{equation}
where $E$ is the elastic modulus, $I$ is the area moment of inertia, and $L_{\textrm{f}}$ is the length of the filament. The area moment for a circular cross-section is $\tfrac{\pi r_{\textrm{f}}^{4}}{4}$, where $r_{\textrm{f}}$ is the radius of the solid filament (for FFF systems the filament diameter is usually 1.75mm or 3mm). So 
\begin{equation}
F_{\textrm{C}} = \frac{\pi^{3} E r_{\textrm{f}}^{4} }{L_{\textrm{f}}^{2}}. 
\end{equation}

In order to avoid buckling, thickening and stiffening filament (which means a high $r_{\textrm{f}}$ and elastic modulus $E$ respectively), and reducing the length of filament between extruder motor and melt zone (reducing $L_{\textrm{f}}$), all help to minimise buckling failures. 

As an order of magnitude example, let's assume that a given printer has a direct drive extruder with a 5mm radius hobbed bolt pushing the solid filament into the nozzle. Then, in this hypothetical example, we read that the stepper motor data sheet says that it can apply a maximum torque of 50 N cm, or a maximum force of the order of 100N using the hobbed bolt with a 5mm radius.  

For 1.75mm diameter PLA filament with an the elastic modulus of 3.5 GPa and 5cm from the melt zone, we find that the critical force is 636N and buckling is unlikely. However, for rubberised and flexible filament, the elastic modulus is 25 MPa (see \cite{filament_discussion}), and so the critical force is 4.5N. Buckling is clearly more likely for this material and can be mitigated by reducing the distance to the melt zone. For example, if the extruder drive to melt zone distance is 1cm rather than 5cm, then the critical force for 1.75mm Ninjaflex is 114N and the stepper motor is less likely to cause filament buckling. 


\section{Extruder designs and drive}

\subsection{Direct drive}

All FFF systems have a motor that forces new filament into the nozzle and molten filament out of the small (typically 0.4mm diameter) hole at the nozzle lower end. Typically the filament is gripped between a hobbed bolt and bearing. A hobbed bolt is shown in Figure \ref{fig:hobbed_bolts}. 

% \begin{figure}[h!!!!!]
% \centering
% \includegraphics[width=0.7\textwidth]{pictures/hobbed_bolts}
% \caption{Six hobbed bolts in which teeth have been cut into shaft of the bolts \cite{hobbed_bolts}.}
% \label{fig:hobbed_bolts}
% \end{figure}

The simplest type of extruder is the direct drive, pictured in Figure \ref{fig:extruder_direct_drive}. This comprises a hobbed bolt on a stepper motor pressing the filament against a bearing. The ridges on the hobbed bolt grip and push the filament through and into the hot end. 

\subsection{Geared drive}

Designs with gears, which subsequently drive the hobbed bolt, have also been developed, which allow finer control of the filament feed rate through the effect of the gears. If, for instance, the stepper motor gear has 10 teeth and the filament drive gear has 43 teeth then ratio of force applied by the drive gear will be $\tfrac{43}{10} = $4.3 times the force applied by the stepper motor \cite{gears}. 

Wade's extruder is a popular example and includes a geared system to turn the hobbed bolt. The herringbone gears provide extra leverage to push the filament into the extruder nozzle when compared to the direct drive version. 


% \begin{figure}[h!!!!!]
% \centering
% \includegraphics[width=0.7\textwidth]{pictures/extruder_direct_drive}
% \caption{A direct drive  extruder in which the hobbed bolt is mounted on the stepper motor's shaft. The filament is gripped between the hobbed bolt and a bearing (taken from \cite{direct_extruder})}
% \label{fig:extruder_direct_drive}
% \end{figure}
% 
% \begin{figure}[h!!!!!]
% \centering
% \includegraphics[width=0.7\linewidth]{pictures/geared_extruder}\\[0.5cm]
% \includegraphics[width=0.7\linewidth]{pictures/geared_extruder_mounted}
% \caption{Top : A geared extruder design (taken from \cite{geared_extruder}) in which the small gear is mounted on the stepper motor shaft which then turns the the larger gear. The hobbed bolt, which in this case has a hexagonal head, is mounted through the centre of the larger gear. The filament can be seen at the top of the picture. Bottom: This picture shows the same extruder design mounted on the printer \cite{geared_extruder}.}
% \label{fig:geared_extruder}
% \end{figure}

The movement of the mass of the extruder and hot end can cause resonances (fundamental vibrations of the frame of the printer), leading to rippling of the surface near features. This effect is shown in \ref{fig:resonances}. 


\section{Alternatives to Filament}

Paste extrusion has been used to print chocolate and conductive paste....

The wire delivery of (Bath MSc)...

\section{Methods to assess the performance of FFF systems}


% \begin{figure}
% 	\centering
% 	\includegraphics[width=0.7\textwidth]{pictures/resonances}
% 	\caption{An example of a surface defect produced by a resonance. Note the rippling which mimics the bar feature (taken from \cite{resonances}). }
% 	\label{fig:resonances}
% \end{figure}

% \begin{thebibliography}{20}
% 	\bibitem{delta_url} The benefits and disadvantages of delta printers are discussed here: \url{https://www.reddit.com/r/3Dprinting/comments/2ahmbp/what_are_the_pros_and_cons_of_the_rostock_delta/} and a discussion of geared and direct drive extruders for delta printers. 
% he
% 	\bibitem{extruders} For a good discussion of the issues involved with different extruder options, please see \url{http://forums.reprap.org/read.php?70,191725}.
% 	
% 	\bibitem{reprap_hotend} The hot end design is discussed at \url{http://reprap.org/wiki/Hot_End_Design_Theory}. 
% 
% 	\bibitem{filament_discussion} A discussion of filaments and their properties \url{http://forums.reprap.org/read.php?41,434821,483991}. 
% 
% 	\bibitem{hobbed_bolts} For a description of how to make a hobbed bolt, please see \url{http://www.thingiverse.com/thing:151669}. 
% 
% 
% 	\bibitem{gears} Discussion of gears and how they affect the applied torque: \url{http://www.maelabs.ucsd.edu/mae_guides/machine_design/machine_design_basics/Mech_Ad/mech_ad.htm}.
% 	
% 	\bibitem{direct_extruder} A direct extruder with a Bowden tube: \url{http://www.thingiverse.com/thing:275593}. 
% 
% 	\bibitem{geared_extruder} An example of a geared extruder is available at \url{http://www.thingiverse.com/thing:961630}. 
% 	
% 	\bibitem{resonances} For a printed object to examine resonances, please see http://www.thingiverse.com/thing:8870. 
% 
% 
% 
% 	\bibitem{nozzle_orifices} Measurements of the lengths of example nozzle orifices: \url{http://jheadnozzle.blogspot.co.uk/2012/05/nozzle-orifice-measurements.html}.  
% 	
% 	
% 	\bibitem{melt_zone} \url{http://3dprinting.stackexchange.com/questions/572/what-is-the-best-length-of-the-melting-zone-in-the-hotend}
% \end{thebibliography}
% 


%----------------------------------------------------------------------------------------

%----------------------------------------------------------------------------------------
%	CHAPTER 4
%----------------------------------------------------------------------------------------
\chapter{Thermal Physics : }
\label{ch:thermal_fdm}
%----------------------------------------------------------------------------------------

\section{Thermal transport background}

\subsection{Thermal Diffusion Equation}

Heat equation and its origins

\subsection{Cylindrical coordinate systems}

Nozzle almost invariably have a cylindrical symmetry so we can express the thermal diffusion equation in this cylindrical coordinate system. 

\section{The Hot End : Applying heat to the filament}




\section{The Heated Bed : Removing temperature gradients}

\section{Thermal Transport Model}

Typically with Fused Filament Fabrication (FFF) techniques a cylindrical cartridge heater is used. For the purposes of this calculation, we need to read the data sheet of the cartridge heater which will give the power $P_{C}$ in Watts (or J/s). From this we can calculate the heat energy flux (the heat energy per unit time per unit area) from the outer cylindrical surface $A_{C}$. We assume that heat energy is lost to the environment when it is radiated out of the circular top and bottom of the cylindrical cartridge heater. The heat energy per unit time per unit area is given by 
\begin{equation}
H_{C} = \frac{P_{C}}{A_{C}}. 
\end{equation}

The heat energy from the cartridge heater is transported to the filament melt zone typically through an aluminium block and then the outer brass cylinder of the extruder nozzle. We simplify this process and assume that the its efficiency is given by $\epsilon$, with typical values of the order of 1 to 5\% level (i.e. 1 to 5\% of the heat energy from the heater cartridge is transported to the melt zone). The amount of heat energy per unit time per unit area entering the melt zone is $\epsilon H_{C}$. 

We now need to find two quantities: Firstly the amount of heat from the cartridge entering the melt zone and secondly the amount of energy required to cause a temperature change. 

In order to calculate the amount of heat entering the melt zone, we need to find the area of the outer surface of the melt zone which is given by $A_{M}$. The heat flux (energy per unit time per area) from the cartridge is $\epsilon H_{C}$ as mentioned above. As this is passing through the outer surface area of the melt zone, we multiply this flux by the melt zone surface area $A_{M}$ to give the rate of energy passing into the melt zone, which is 
\begin{equation}
P_{M} = \frac{\epsilon P_{C} A_{M} }{A_{C}}
\label{H}
\end{equation}
and gives the heat energy per unit time (in Watts) passing into the melt zone. 

The second quantity, that we need to estimate, is the effect of this heat transport passing into the melt zone. This heat energy raises the temperature of the filament melt. The energy required to do this per temperature rise (in Kelvin) per mass of melt (in kg) is the specific heat capacity $C_{p}$ typically expressed in units of J K$^{-1}$ kg$^{-1}$. 

So now we need to find the mass of the melt (in kg) to find how much energy is required to raise the temperature of the whole of the melt zone . 

The mass of the filament melt is melt volume multiplied by its density $\rho$, or 
\begin{equation}
m = \rho V_{M}. 
\end{equation}
The heat capacity (the energy required for a unit temperature change) is then 
\begin{equation}
c = C_{p} \rho V_{M} 
\label{C}
\end{equation}
which gives the amount of energy required for a temperature rise of 1~K. 

So in equation \ref{H} we have the amount of energy per second passing into the melt zone, and in equation \ref{C} we have the amount of energy required for a 1~K temperature rise in the melt zone.  In order to find the rate of temperature rise $\frac{dT}{dt}$ we divide equation \ref{H} by equation \ref{C} to give 
\begin{equation}
\frac{dT}{dt} = \frac{\epsilon P_{C} A_{M} }{\rho A_{C} C_{P} V_{M}}. 
\label{T}
\end{equation}

Just to repeat: I've used the convention that quantities with a subscript of $C$ refer to the Cartridge heater, and ones with a subscript of $M$ refer to the melt zone. 

\section{A Worked Example}

Here we work through this to calculate the expected order of magnitude rate of temperature change in the melt zone. 

With a cylindrical melt zone of 1.8mm diameter and 17mm high, the outer surface area (excluding the end caps) is $A_{M} = 96$~mm$^{2}$ and the volume is $V_{M} = 43$~mm$^{3}$. 

\begin{figure}[h!!!]
\centering
\includegraphics[width=0.7\linewidth]{pictures/12V-40W-Reprap-Cartridge-Heater.png}
\caption{A typical 3D FFF printer heater cartridge.}
\label{fig:12V-40W-Reprap-Cartridge-Heater}
\end{figure}

\begin{figure}[h!!!]
\centering
\includegraphics[width=0.7\linewidth]{pictures/nozzle_assembled}
\caption{An assembled nozzle with heater cartridge inserted into the aluminium heat transfer block. This particular nozzle also has a Bowden tube attached.}
\label{fig:nozzle_assembled}
\end{figure}


The heater cartridge has a power of $P_{C} = 40$~W and cylindrical dimensions of 6~mm in diameter, and 20~mm in height. So the heater outer surface area is $A_{C} = 377$~mm$^{2}$. 

We assume that the heat transport efficiency $\epsilon$ is 1~\%. 

We also assume that the filament material is PLA with a density of $\rho = 1.25$~g/cm$^{3}$ and specific heat capacity of $C_{P} = $1800~J K$^{-1}$ kg$^{-1}$. 

Note that the density is in units of g/cm$^{3}$ and the other quantities are in units of kg and mm, so let's convert the density to kg/mm$^{-3}$: There is 10$^{-3}$~kilograms per gram (from the SI definitions) so $\rho = 1.25 \times 10^{-3}$~kg/cm$^{3}$; then 0.1 cm is the same as 1 mm, or 10$^{-3}$ cm$^{3}$ per mm$^{3}$. This eventually gives $\rho = 1.25 \times 10^{-6}$~kg/mm$^{3}$


The temperature rise is hen 
\begin{equation}
\frac{dT}{dt} = \frac{0.01 \times 40 \times 96 }{\left( 1.25 \times 10^{-6}\right) \times 377 \times 1800 \times 43}. 
\end{equation}
or $\frac{dT}{dt} = 38.4/36.5 = 1.1 K/s$.  

So the time taken to change temperature from the 20~$^{\circ}$C to 210~$^{\circ}$C, which is a temperature change of 190~K, is given by $190 / 1.1 = 180$~s. 

From experience this is towards the upper end of the time taken to warm the nozzle up. 

If the heat transport efficiency is 5~\%, then the rate of temperature rise increases by a factor of five and the time taken reduces by the same factor. The time taken is then 36~s.  

It is also possible to appreciate the effect of changing filament thickness to the other standard diameter of 3~mm from 1.8~mm. This increases the surface area by a factor of $3/1.8 = 1.67$ and the volume by a factor of $\left( 3/1.8 \right)^{2} = 2.78$. As the rate of temperature rise (see equation \ref{T}) is proportional to the ratio of melt zone area to melt zone volume, then the it will decrease by a factor of $1.67 / 2.78$ or 0.6. So, if all other parameters are kept constant, then the rate of temperature rise is 0.6 times 1.1 K/s = 0.66 K/s and the time taken to heat up is also proportionately longer (between 55~s and 272~s for $\epsilon$ of 1~\% and 5~\% respectively). 













%


\section{Thermal Transport Model}

Typically with Fused Filament Fabrication (FFF) techniques a cylindrical cartridge heater is used. For the purposes of this calculation, we need to read the data sheet of the cartridge heater which will give the power $P_{C}$ in Watts (or J/s). From this we can calculate the heat energy flux (the heat energy per unit time per unit area) from the outer cylindrical surface $A_{C}$. We assume that heat energy is lost to the environment when it is radiated out of the circular top and bottom of the cylindrical cartridge heater. The heat energy per unit time per unit area is given by 
\begin{equation}
H_{C} = \frac{P_{C}}{A_{C}}. 
\end{equation}

The heat energy from the cartridge heater is transported to the filament melt zone typically through an aluminium block and then the outer brass cylinder of the extruder nozzle. We simplify this process and assume that the its efficiency is given by $\epsilon$, with typical values of the order of 1 to 5\% level (i.e. 1 to 5\% of the heat energy from the heater cartridge is transported to the melt zone). The amount of heat energy per unit time per unit area entering the melt zone is $\epsilon H_{C}$. 

We now need to find two quantities: Firstly the amount of heat from the cartridge entering the melt zone and secondly the amount of energy required to cause a temperature change. 

In order to calculate the amount of heat entering the melt zone, we need to find the area of the outer surface of the melt zone which is given by $A_{M}$. The heat flux (energy per unit time per area) from the cartridge is $\epsilon H_{C}$ as mentioned above. As this is passing through the outer surface area of the melt zone, we multiply this flux by the melt zone surface area $A_{M}$ to give the rate of energy passing into the melt zone, which is 
\begin{equation}
P_{M} = \frac{\epsilon P_{C} A_{M} }{A_{C}}
\label{H}
\end{equation}
and gives the heat energy per unit time (in Watts) passing into the melt zone. 

The second quantity, that we need to estimate, is the effect of this heat transport passing into the melt zone. This heat energy raises the temperature of the filament melt. The energy required to do this per temperature rise (in Kelvin) per mass of melt (in kg) is the specific heat capacity $C_{p}$ typically expressed in units of J K$^{-1}$ kg$^{-1}$. 

So now we need to find the mass of the melt (in kg) to find how much energy is required to raise the temperature of the whole of the melt zone . 

The mass of the filament melt is melt volume multiplied by its density $\rho$, or 
\begin{equation}
m = \rho V_{M}. 
\end{equation}
The heat capacity (the energy required for a unit temperature change) is then 
\begin{equation}
c = C_{p} \rho V_{M} 
\label{C}
\end{equation}
which gives the amount of energy required for a temperature rise of 1~K. 

So in equation \ref{H} we have the amount of energy per second passing into the melt zone, and in equation \ref{C} we have the amount of energy required for a 1~K temperature rise in the melt zone.  In order to find the rate of temperature rise $\frac{dT}{dt}$ we divide equation \ref{H} by equation \ref{C} to give 
\begin{equation}
\frac{dT}{dt} = \frac{\epsilon P_{C} A_{M} }{\rho A_{C} C_{P} V_{M}}. 
\label{T}
\end{equation}

Just to repeat: I've used the convention that quantities with a subscript of $C$ refer to the Cartridge heater, and ones with a subscript of $M$ refer to the melt zone. 

\section{A Worked Example}

Here we work through this to calculate the expected order of magnitude rate of temperature change in the melt zone. 

With a cylindrical melt zone of 1.8mm diameter and 17mm high, the outer surface area (excluding the end caps) is $A_{M} = 96$~mm$^{2}$ and the volume is $V_{M} = 43$~mm$^{3}$. 

\begin{figure}[h!!!]
\centering
\includegraphics[width=0.7\linewidth]{pictures/12V-40W-Reprap-Cartridge-Heater.png}
\caption{A typical 3D FFF printer heater cartridge.}
\label{fig:12V-40W-Reprap-Cartridge-Heater}
\end{figure}

\begin{figure}[h!!!]
\centering
\includegraphics[width=0.7\linewidth]{pictures/nozzle_assembled}
\caption{An assembled nozzle with heater cartridge inserted into the aluminium heat transfer block. This particular nozzle also has a Bowden tube attached.}
\label{fig:nozzle_assembled}
\end{figure}


The heater cartridge has a power of $P_{C} = 40$~W and cylindrical dimensions of 6~mm in diameter, and 20~mm in height. So the heater outer surface area is $A_{C} = 377$~mm$^{2}$. 

We assume that the heat transport efficiency $\epsilon$ is 1~\%. 

We also assume that the filament material is PLA with a density of $\rho = 1.25$~g/cm$^{3}$ and specific heat capacity of $C_{P} = $1800~J K$^{-1}$ kg$^{-1}$. 

Note that the density is in units of g/cm$^{3}$ and the other quantities are in units of kg and mm, so let's convert the density to kg/mm$^{-3}$: There is 10$^{-3}$~kilograms per gram (from the SI definitions) so $\rho = 1.25 \times 10^{-3}$~kg/cm$^{3}$; then 0.1 cm is the same as 1 mm, or 10$^{-3}$ cm$^{3}$ per mm$^{3}$. This eventually gives $\rho = 1.25 \times 10^{-6}$~kg/mm$^{3}$


The temperature rise is hen 
\begin{equation}
\frac{dT}{dt} = \frac{0.01 \times 40 \times 96 }{\left( 1.25 \times 10^{-6}\right) \times 377 \times 1800 \times 43}. 
\end{equation}
or $\frac{dT}{dt} = 38.4/36.5 = 1.1 K/s$.  

So the time taken to change temperature from the 20~$^{\circ}$C to 210~$^{\circ}$C, which is a temperature change of 190~K, is given by $190 / 1.1 = 180$~s. 

From experience this is towards the upper end of the time taken to warm the nozzle up. 

If the heat transport efficiency is 5~\%, then the rate of temperature rise increases by a factor of five and the time taken reduces by the same factor. The time taken is then 36~s.  

It is also possible to appreciate the effect of changing filament thickness to the other standard diameter of 3~mm from 1.8~mm. This increases the surface area by a factor of $3/1.8 = 1.67$ and the volume by a factor of $\left( 3/1.8 \right)^{2} = 2.78$. As the rate of temperature rise (see equation \ref{T}) is proportional to the ratio of melt zone area to melt zone volume, then the it will decrease by a factor of $1.67 / 2.78$ or 0.6. So, if all other parameters are kept constant, then the rate of temperature rise is 0.6 times 1.1 K/s = 0.66 K/s and the time taken to heat up is also proportionately longer (between 55~s and 272~s for $\epsilon$ of 1~\% and 5~\% respectively). 









%----------------------------------------------------------------------------------------
%	CHAPTER 5
%----------------------------------------------------------------------------------------

\chapter{Material Physics : }
\label{ch:materials_fdm}

\section{Materials Physics for Fused Deposition Modelling}

Feedstock materials for FDM systems are very varied plastics. PLA and ABS are extremely common within the community and we'll discuss them in more detail here. 

\section{Polymers}

Polymers are molecules with long backbone of carbon atoms. Carbon, being a group IV element, may have up to four single electrons in order to form bonds with neighbours\sidenote{Although it's out of the scope of this book, it's important to understand that materials form regular structures by sharing electrons to some extent: partially, termed a covalent bond, or fully through an ionic bond (such as Salt, written as Na$^{+}$Cl$^{-}$ with the indications of charge), or through hydrogen bonds, in which the hydrogen attains a slight positive charge attracting negatively charged molecules nearby.}. Hydrogen can form these bonds quite easily for the majority of carbon atoms and, when Hydrogen atoms are attached to all Carbon atoms, then you have a molecule of Polyethylene. 


\newcommand\setpolymerdelim[2]{\def\delimleft{#1}\def\delimright{#2}}

\def\makebraces[#1,#2]#3#4#5{%
\edef\delimhalfdim{\the\dimexpr(#1+#2)/2}%
\edef\delimvshift{\the\dimexpr(#1-#2)/2}%
\chemmove{%
\node[at=(#4),yshift=(\delimvshift)]
{$\left\delimleft\vrule height\delimhalfdim depth\delimhalfdim width0pt\right.$};%
\node[at=(#5),yshift=(\delimvshift)]
{$\left.\vrule height\delimhalfdim depth\delimhalfdim width0pt\right\delimright_{\rlap{$\scriptstyle#3$}}$};}}

\setpolymerdelim() 

\begin{fullwidth}
\begin{table}
\begin{tabular}{cc}
\hline \hline Name & Chemical Formula \\[0.5cm]
\bigskip Polyethylene & \chemfig{\vphantom{CH_2}-[@{op,.75}]CH_2-CH_2-[@{cl,0.25}]} \makebraces[5pt,5pt]{\!\!n}{op}{cl} \bigskip \\
\bigskip Polyvinyl chloride & \chemfig{\vphantom{CH_2}-[@{op,.75}]CH_2-CH(-[6]Cl)-[@{cl,0.25}]} \makebraces[5pt,25pt]{\!\!\!n}{op}{cl} \bigskip \\
\bigskip Nylon 6 & \chemfig{\phantom{N}-[@{op,.75}]{N}(-[2]H)-C(=[2]O)-{(}CH_2{)_5}-[@{cl,0.25}]} \makebraces[30pt,5pt]{}{op}{cl} \bigskip \\



\bigskip Polylactide Acid & \chemfig{\vphantom{C}-[@{op,.5}]{O}-{CH}(-[2]CH_3)-{C}(=[2]O)-[@{cl,0.5}]} \makebraces[25pt,5pt]{\!\!\!n}{op}{cl} \bigskip \\



\bigskip Polyethylene Terapthalate & \chemfig{-O-{C}(=[2]O)-[:-0]*6(-=-(-{C}(=[2]O)-O-CH_2-CH_2-)=-=-)} \makebraces[25pt,5pt]{\!\!\!n}{op}{cl} \bigskip \\





\hline \hline 
\end{tabular}
\caption{A number of common polymers and their chemical formulae.}
\end{table}
\end{fullwidth}


and have a wide variety of material behaviour 

PLA and ABS are branched polymers 

\section{Stress and Strain}

Stress and strain describe the causes of material's deformation (applied forces)and their effect (the amount of deformation; stress is the force applied to a material's surface per unit area and strain is the extension per unit original length. 

Stress and strain are vector quantities and it's important to be clear which \emph{type} of stress is being discussed. 

The behaviour of materials can be characterised by its stress strain curve in which there are a number of different regions of behaviour. For small stresses and strains, the material responds linearly and the proportionality constant is given the name Young's modulus (or Elastic modulus). 

A material with a small Youngs modulus will deform very little for a given applied force, and the material could be described as stiff or rigid. Materials with a behaviour similar to soft rubber will deform more easily so these will have small Youngs moduli. 

\begin{table}[h]\index{Example Youngs moduli of different AM materials}
\footnotesize%
\begin{center}
\begin{tabular}{cc}
\toprule
Material & Youngs Modulus (GPa) \\
\midrule
PLA & \\
ABS & \\
\bottomrule
\end{tabular}
\end{center}
\caption{The measured Youngs moduli for FDM materials. }
\label{tab:youngs_moduli}
\end{table}


\section{Phase Transitions}

The phase of a material describes its state and, as heat is applied to the filament within the melt zone, undergoes a transition from solid, through its glass transition, and then to a liquid. 


\section{Viscous Flow}

Viscosity is the property of fluids to experience (shear) strain when it is flowing past a surface. 


%----------------------------------------------------------------------------------------
%	CHAPTER 6
%----------------------------------------------------------------------------------------

\chapter{Electronics}
\label{ch:hardware}


\chapter{3D Scanning for model generation}
\label{ch:model_generation}
% \include{}






\backmatter

\chapter{Glossary}

\begin{itemize}
	\item[\textbf{Intensity}] The number of photons incident on the surface of the \SL~resin \emph{per unit time per unit area}. This is the same as the rate of photons per unit area. 
	
	\item[\textbf{Exposure}] The total (time-integrated) number of photons per unit area that have been absorbed by the resin in a short printing run. 
			
	For example, if the laser intensity is assumed to be constant at 10$^{5}$ photons cm$^{-2}$ $s^{-1}$ for 30 seconds then the exposure is 10$^{5}$ times 30 seconds, or 3$\times $10$^{6}$ photons cm$^{-2}$.  
	
	\item[\textbf{Critical Exposure}] The minimum number of photons per unit area required for curing to occur in that region. 
	 
	\item[\textbf{Penetration Depth}] A property of the resin describing how attenuating the resin is at the laser's wavelength. 	
	 
\end{itemize}




%----------------------------------------------------------------------------------------
%	BIBLIOGRAPHY
%----------------------------------------------------------------------------------------

\bibliography{bibliography} % Use the bibliography.bib file for the bibliography
\bibliographystyle{plainnat} % Use the plainnat style of referencing

%----------------------------------------------------------------------------------------

\printindex % Print the index at the very end of the document

\end{document}