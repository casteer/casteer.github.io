
\section{Materials Physics for Fused Deposition Modelling}

Feedstock materials for FDM systems are very varied plastics. PLA and ABS are extremely common within the community and we'll discuss them in more detail here. 

\section{Polymers}

Polymers are molecules with long backbone of carbon atoms. Carbon, being a group IV element, may have up to four single electrons in order to form bonds with neighbours\sidenote{Although it's out of the scope of this book, it's important to understand that materials form regular structures by sharing electrons to some extent: partially, termed a covalent bond, or fully through an ionic bond (such as Salt, written as Na$^{+}$Cl$^{-}$ with the indications of charge), or through hydrogen bonds, in which the hydrogen attains a slight positive charge attracting negatively charged molecules nearby.}. Hydrogen can form these bonds quite easily for the majority of carbon atoms and, when Hydrogen atoms are attached to all Carbon atoms, then you have a molecule of Polyethylene. 


\newcommand\setpolymerdelim[2]{\def\delimleft{#1}\def\delimright{#2}}

\def\makebraces[#1,#2]#3#4#5{%
\edef\delimhalfdim{\the\dimexpr(#1+#2)/2}%
\edef\delimvshift{\the\dimexpr(#1-#2)/2}%
\chemmove{%
\node[at=(#4),yshift=(\delimvshift)]
{$\left\delimleft\vrule height\delimhalfdim depth\delimhalfdim width0pt\right.$};%
\node[at=(#5),yshift=(\delimvshift)]
{$\left.\vrule height\delimhalfdim depth\delimhalfdim width0pt\right\delimright_{\rlap{$\scriptstyle#3$}}$};}}

\setpolymerdelim() 

\begin{fullwidth}
\begin{table}
\begin{tabular}{cc}
\hline \hline Name & Chemical Formula \\[0.5cm]
\bigskip Polyethylene & \chemfig{\vphantom{CH_2}-[@{op,.75}]CH_2-CH_2-[@{cl,0.25}]} \makebraces[5pt,5pt]{\!\!n}{op}{cl} \bigskip \\
\bigskip Polyvinyl chloride & \chemfig{\vphantom{CH_2}-[@{op,.75}]CH_2-CH(-[6]Cl)-[@{cl,0.25}]} \makebraces[5pt,25pt]{\!\!\!n}{op}{cl} \bigskip \\
\bigskip Nylon 6 & \chemfig{\phantom{N}-[@{op,.75}]{N}(-[2]H)-C(=[2]O)-{(}CH_2{)_5}-[@{cl,0.25}]} \makebraces[30pt,5pt]{}{op}{cl} \bigskip \\



\bigskip Polylactide Acid & \chemfig{\vphantom{C}-[@{op,.5}]{O}-{CH}(-[2]CH_3)-{C}(=[2]O)-[@{cl,0.5}]} \makebraces[25pt,5pt]{\!\!\!n}{op}{cl} \bigskip \\



\bigskip Polyethylene Terapthalate & \chemfig{-O-{C}(=[2]O)-[:-0]*6(-=-(-{C}(=[2]O)-O-CH_2-CH_2-)=-=-)} \makebraces[25pt,5pt]{\!\!\!n}{op}{cl} \bigskip \\





\hline \hline 
\end{tabular}
\caption{A number of common polymers and their chemical formulae.}
\end{table}
\end{fullwidth}


and have a wide variety of material behaviour 

PLA and ABS are branched polymers 

\section{Stress and Strain}

Stress and strain describe the causes of material's deformation (applied forces)and their effect (the amount of deformation; stress is the force applied to a material's surface per unit area and strain is the extension per unit original length. 

Stress and strain are vector quantities and it's important to be clear which \emph{type} of stress is being discussed. 

The behaviour of materials can be characterised by its stress strain curve in which there are a number of different regions of behaviour. For small stresses and strains, the material responds linearly and the proportionality constant is given the name Young's modulus (or Elastic modulus). 

A material with a small Youngs modulus will deform very little for a given applied force, and the material could be described as stiff or rigid. Materials with a behaviour similar to soft rubber will deform more easily so these will have small Youngs moduli. 

\begin{table}[h]\index{Example Youngs moduli of different AM materials}
\footnotesize%
\begin{center}
\begin{tabular}{cc}
\toprule
Material & Youngs Modulus (GPa) \\
\midrule
PLA & \\
ABS & \\
\bottomrule
\end{tabular}
\end{center}
\caption{The measured Youngs moduli for FDM materials. }
\label{tab:youngs_moduli}
\end{table}


\section{Phase Transitions}

The phase of a material describes its state and, as heat is applied to the filament within the melt zone, undergoes a transition from solid, through its glass transition, and then to a liquid. 


\section{Viscous Flow}

Viscosity is the property of fluids to experience (shear) strain when it is flowing past a surface. 
