\section{\SL Process Description}

[Picture of the \SL system]

\SL was developed in the mid-80s by **** working for the ****. 

The components of a \SL system comprise a vat of photo-sensitive gel, a laser and a motorised platform. \SL prints objects by layers and, for each layer of the printed item, the laser moves to each point in that layer which is required to be solid. The laser initiates a polymerisation processes which results in the gel's solidfication near the laser. 


\section{Laser Properties}


Within a \SL system the laser initiates the curing process and is specified by its power and wavelength. For the purposes of modelling the \SL process we need to find out how to calculate the rate of photons hitting the upper surface of the resin from these specifications. 

\subsection{How much energy is carried by a single photon?} 

Here we calculate how much energy is transmitted by a single photon from the \SL laser system. 

Formlabs are a leading company that produce \SL printers so as an example let's use their specification of the Formlabs One \SL printer. The Formlabs Two \SL printer has a 405nm Violet laser with 250 mW maximum power \sidenote{For the specification of the Formlabs two, please see: http://formlabs.com/products/3d-printers/tech-specs/}. 

So, in order to calculate the energy carried by a single photon of wavelength $\lambda$, we first use the relationship
\begin{equation}
E = \frac{hc}{\lambda}, 
\end{equation}
where $h$ is Planck's constant and $c$ is the speed of light. So putting in the numbers gives $E = \frac{6.63 \times 10^{-34} \times 3 \times 10^{8} }{6.63 \times 10^{-9}} = 4.9 \times 10^{-19}$ J. 

It is sometimes conventional to use the units of electron-volts\sidenote{The electron volt is defined as the energy given to an electron by an accelerating potential of 1V and is given the synbole eV.} rather than the SI unit of joules (whose symbol is J) when talking about energies. In order to convert between them we divide by the charge of an electron, or $1.6 \times 10^{-19}$ C. The energy of a single photon from a 405nm laser is then $4.9 \times 10^{-19} / 1.6 \times 10^{-19} = 3.1$eV. 

Why does this matter? The solidification process is initiated by the laser photons and requires the breaking of bonds within the gel's polymers. These bonds cannot be broken by photons of lower energies, or longer wavelengths, such as those in the infra-red (500nm and upwards). If we were to try to use an infra-red laser then we'd find that the gel is transparent for this wavelength. This particular property is exploited in the two-photon polymerisation printing technique which is described later in section ***. 

\subsection{How many photons are delivered per second by a typical Stereolithography laser?} 

The speed at which the initiation occurs over the laser's spot region is governed by the rate photons hiting this region. So we're also able to estimate this: Returning to the original Formlabs two specification as our illustrative example, we would like to calculate the laser intensity from its power specification of 250 mW. The maximum laser power is 250mW\sidenote{Watts are the derived SI unit for power, which is given in joules per second.} or $250 \times 10^{-3}$ J/s; this is the maximum amount of energy transferred by the photons in the laser's beam. As we've already found that the energy per photon is $4.9 \times 10^{-19}$~J so the rate of photons is $250 \times 10^{-3} / 4.9 \times 10^{-19} = 5.1 \times 10^{17}$~photons per sec. 

[Summary of calculation methods?]

\section{Laser Intensity and its Attenuation in Resin}

The photons in the laser beam enter the resin perpendicularly at its surface and travel through the resin until they interact with a polymer and start the solidification process. As the photons travel further into the gel and it has passed by more resin, it is increasingly likely that more and more become absorbed by the resin. We model this by expressing the fractional loss of the intensity \sidenote{We define the photon beam intensity as the number of photons per unit area of the gel's surface, per unit time.} of photons is proportional to the distance travelled through the resin. This is also called the Beer-Lambert law and can be expressed as 
\begin{equation}
\frac{- dN}{N} \propto dz
\end{equation}
where a small number $dN$ of photons have been lost (hence the negative sign) in a small distance $dz$; we take the $z$-axis as the direction downwards into the resin starting at $z=0$ at the resin's surface. 

We then set the proportonality constant to be $1 / D_{p}$, where $D_{p}$ is called the penetration depth, and so we can then write 
\begin{equation}
\frac{- dN}{N} = \frac{dz}{D_{p}}. 
\end{equation}

At the resin's surface, $z=0$, the laser intensity is $N_{0}$ (this is a constant, and we will find out later how to relate this to the laser power), The laser intensity decreases to $N\left(z\right)$ at a distance $z$ into the resin (which we would like to find out how this varies). 

Integrating both sides gives 
\begin{equation}
\int_{N_{0}}^{N\left(z\right)} \frac{- dN}{N} = \int_{0}^{z} \frac{dz}{D_{p}}. 
\label{N0}
\end{equation}
The left-hand side of equation \ref{N0} is 
\begin{eqnarray}
\int_{N_{0}}^{N\left(z\right)} \frac{- dN}{N} & = & \ln \left[ N\left(z\right)\right]  - \ln \left[ N_{0} \right]. \\
& = & \ln \left[ \frac{N\left(z\right)}{ N_{0} } \right]. \\
\end{eqnarray}
The right-hand side of equation \ref{N0} is just $\frac{z}{D_{p}}$. 

Equating the left and right hand sides gives 
\begin{equation}
 - \ln \left[ \frac{N\left(z\right)}{ N_{0} } \right] = \frac{z}{D_{p}}. 
\end{equation}
The next steps then take the negative sign over to the right hand side of this equation and take the exponential of both sides. This then results in 
\begin{equation}
\frac{N\left(z\right)}{ N_{0} } = \exp \left\{ - \frac{z}{D_{p}}. \right\}
\end{equation}
and then finally the main result: 
\begin{equation}
N\left(z\right)  = N_{0} \exp \left\{ - \frac{z}{D_{p}}. \right\}
\end{equation}

The physical significance of the penetration depth is that it is the distance after which the laser beam has lost $\exp \left\{ -1 \right\}$, or 37\%, of its intensity. The penetration depth is a property of the choice of resin and expresses the degree to which the resin attenuates the laser beam. 

For example, at a distance of 1mm into the resin, the intensity falls to a fraction of $\frac{N\left( z = \textrm{1 mm} \right)}{ N_{0} } =  \exp \left\{ - \frac{1}{D_{p}} \right\}$, where $D_{p}$ is also measured in mm. So for a highly-attenuating resin with a penetration depth of 0.1mm, we find the beam has fallen to $\exp \left\{ - \frac{1}{0.1} \right\} = 0.000045$ of its intensity at the surface of the resin. For a poorly attenuating resin, let's say that $D_{p} = 10$mm, the intensity, at 1mm into the resin, has fallen by $\exp \left\{ - \frac{1}{10} \right\} = 0.90$, or 90\% of its incident intensity at the surface.\\[1cm] 


\section{The solidfication region}

A common assumption to make is that the laser spot at the gel surface ($z=0$) has a Gaussian spread, so the intensity perpendicular to the beam is given by
\begin{equation}
 N\left(x,y,0 \right) = N\left(x=0,y=0,0 \right) \exp\left( - 2\frac{x^{2}+y^{2}}{\sigma_{r}} \right) 
\end{equation}
where the parameter $\sigma_{r}$ determines the spread of the beam. 

The Gaussian laser beam has cylindrical symmetry - it only depends on $r^{2} = x^{2}+y^{2}$ and it is often more convenient to write this as 
\begin{equation}
 N\left(r, 0 \right) = N\left(r=0, 0 \right) \exp\left( - \frac{r^{2}}{\sigma_{r}^{2}} \right) 
\end{equation}

The photon intensity over all spatial coordinates is then 
\begin{equation}
N\left(r,z\right)  = N\left(r=0,z=0\right) \exp \left\{ - \frac{z}{D_{p}} - \frac{r^{2}}{\sigma_{r}^{2}}  \right\}
\end{equation}

\subsection{Total laser power}

The total laser power\sidenote{We define the laser power here as the number of photons per unit time incident on the whole gel surface.} is given by integrating over the directions perpendicular to the laser's direction. Using cylindrical coordinates, the power integral is 
\begin{eqnarray}
 P & = & N\left(r=0, 0 \right) \int_{0}^{2\pi} \int_{0}^{\infty} N\left(r, 0 \right) r dr d\theta \\
   & = & N\left(r=0, 0 \right) \int_{0}^{2\pi} \int_{0}^{\infty} \exp\left( - \frac{r^{2}}{\sigma_{r}^{2}} \right) r dr d\theta \\
   & = & N\left(r=0, 0 \right) 2\pi \int_{0}^{\infty} \exp\left( - \frac{r^{2}}{\sigma_{r}^{2}} \right) r dr \\
   & = & N\left(r=0, 0 \right) \pi \sigma_{r}^{2} \int_{0}^{\infty} \exp\left( - u  \right) du \\
   & = & N\left(r=0, 0 \right) \pi \sigma_{r}^{2}
\end{eqnarray}
where $\theta$ is the angular coordinate, and the integral's variables were transformed by $u =  \frac{r^{2}}{\sigma_{r}}$. We can express the intensity of photons at the centre of the beam in terms of its power by a rearrangement of this
\begin{equation}
   N\left(r=0, 0 \right)  = \frac{P}{\pi \sigma_{r}^{2}}
\end{equation}

[Think we are missing the energy per photon as a factor in front of the expression]

\subsection{Total laser exposure and the shape of the solidified region }

The laser exposure is the total number of photons per unit area over a period of time\sidenote{The exposure is defined as the total number of photons that have been incident on the gel surface per unit area}. 

It's also important to consider the laser moving across the surface of the gel in a radial direction at a speed $v_{r}$
\begin{equation}
 E\left(x,y,z\right) = \frac{P}{\pi \sigma_{r}^{2}}  \int_{0}^{t} \exp \left\{ - \frac{z}{D_{p}} - \frac{x^{2}+y^{2}}{\sigma_{r}^{2}}  \right\} dt 
\end{equation}

Changing coordinates from the time to radial coordinate in the integral to 
\begin{equation}
 v_{x} = \frac{dx}{dt}
\end{equation}
and then 
\begin{equation}
 E\left(x,y,z\right) = \frac{Pv_{r}}{\pi \sigma_{r}^{2}}  \exp \left\{ - \frac{z}{D_{p}}\right\} \int_{0}^{v_{x} t}  \exp \left\{ - \frac{x^{2}+y^{2}}{\sigma_{r}^{2}}  \right\} dx 
\end{equation}


\begin{equation}
 E\left(y,z\right) = \frac{P v_{r}}{\pi \sigma_{r}^{2}}  \exp \left\{ - \frac{z}{D_{p}}\right\} \exp \left\{ - \frac{y^{2}}{\sigma_{r}^{2}}  \right\} \int_{0}^{v_{x} t}  \exp \left\{ - \frac{x^{2}}{\sigma_{r}^{2}}  \right\} dx
\end{equation}

To tidy up this expression, we write 
\begin{equation}
 E\left(y,z\right) = E_{max}  \exp \left\{ - \frac{z}{D_{p}}\right\} \exp \left\{ - \frac{y^{2}}{\sigma_{r}^{2}}  \right\} 
\end{equation}
where 
\begin{equation}
 E_{max} = \frac{P v_{r}}{\pi \sigma_{r}^{2}} \int_{0}^{v_{x} t}  \exp \left\{ - \frac{x^{2}}{\sigma_{r}^{2}}  \right\} dx. 
 \label{emax}
\end{equation}

As the integral does not depend on the $y$- or $z-$ coordinates, the shape of the solidified region is given by the factor in front of the integral. 
\begin{equation}
 E\left(y,z\right)  = E_{max} \exp \left\{ - \frac{z}{D_{p}}\right\} \exp \left\{ - \frac{y^{2}}{\sigma_{r}^{2}}  \right\} 
\end{equation}

The cured, or solid, region is where the exposure $E\left(y,z\right)$ is greater than a critical exposure $E_{c}$. The shape of the cured region is 
\begin{equation}
 E\left(y,z\right)  = E_{c} = E_{max} \exp \left\{ - \frac{z}{D_{p}}\right\} \exp \left\{ - \frac{y^{2}}{\sigma_{r}^{2}}  \right\} 
\end{equation}
or rearranging 
\begin{equation}
\ln \left( \frac{ E_{max}}{E_{c} } \right) =  \frac{z}{D_{p}} + \frac{y^{2}}{\sigma_{r}^{2}} 
\label{shape}
\end{equation}
which is a parabolic cylinder facing down from the gel's surface. 

\subsection{The cure depth}

The cure depth $C_{d}$ is the depth to which the solidified region extends and, from equation \ref{shape}, this maximum is at $y=0$. 
\begin{equation}
\ln \left( \frac{ E_{max}}{E_{c} } \right) =  \frac{C_{d}}{D_{p}} 
\label{cd}
\end{equation}
It is possible to control the layer height by selecting a gel resin with the right $E_{c}$ and $D_{p}$, control the laser power, which is related to $E_{max}$ through equation \ref{emax}. 

This allows a measurement of the penetration depth by varying the maximum exposure $E_{max}$, through the laser power and time, and plotting $\ln \left( E_{max} \right)$ against the measured cure depth; from equation \ref{cd} the gradient of this plot is $D_{p}$ and the intercept gives $E_{c}$. 

[example plot]

[worked example?]

\subsection{The linewidth}

The resolution of Stereolithographic prints can also be found from equation \ref{shape}. As the printer is usually concerned with the smallest feature that is possible, the finest resolution is limited by the widest point of the cured region. The widest point is at the surface, when $z=0$, and is 
\begin{eqnarray}
L_{w} & = & 2 \sigma_{r} \sqrt{ \ln \left( \frac{ E_{max}}{E_{c} } \right) } \\
L_{w} & = & 2 \sigma_{r} \sqrt{ \frac{C_{d}}{D_{p}} } \\
\label{lw}
\end{eqnarray}
So if one wanted to produce very small feature sizes then the printer would choose a small laser spot, and a material with a high penetration depth and small cure depth. 








