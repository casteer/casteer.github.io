% \documentclass[a4paper,10pt]{article}
% \usepackage[utf8]{inputenc}
% 
% %opening
% \title{Social Impact of Additive Manufacturing}
% \author{C. A. Steer}
% 
% \begin{document}
% 
% \maketitle
% 
% \begin{abstract}
% \end{abstract}

% http://www.theatlantic.com/magazine/archive/2000/03/the-kept-university/306629/

\section{Intellectual property and AM}

Intellectual property \sidenote{For the ideallist, this is the oddest chapter - an aspect of 3D printing design that I hope that you'll never have to use regularly and is dry in its pure lack of physics. \emph{But}, if you'd like to be credited in a particular manner or would like to know how cite your new friends in the 3D printing community properly (so that they become lifelong friends), then you need to know about the basics of intellectual property. Don't take too long to read this chapter, skim it and know that it's here to help but generally I'd move on to the good stuff elsewhere in this book.} comprises the creative and innovative inventions of designers or technologists and important to understand regardless of whether you are designing for personal or commercial reasons. 

When designing for personal enjoyment you may wish to provide your design back to the community and, especially if it's based on someone else's design, then you need to be aware of both the right way to credit your inspiration, or even how you'd like to cited, and the implications of their or your own choice of licensing\sidenote{Thingiverse, see \url{www.thingiverse.com}, is a popular website to upload your 3D printed designs and deals with intellectual property in an interesting manner. The user has a selection of licenses to choose from, including a variety of Creative Commons licenses, giving details on how you'd like to cited, or have strange acronyms such as BSD, GPL or LGPL. Please see section \ref{licensing} for more details on these.}

\subsection{Copyright Protection}

Copyright, a form of intellectual property, refers to ideas within a medium so, for example, this could be literary, musical or another artistic endeavour. To copyright a item, the designer can place a copyright notice on the item such as `Copyright 2016 \copyright Joe Bloggs`, or even leave it off as copyright is applied from the moment that the thing is created\sidenote{In the UK, a good website to start researching into copyright is: \url{https://www.gov.uk/copyright/overview}}. Copyright means that others cannot copy or distribute your work. 

\subsection{Patent Protection}

Patents are another form of intellectual property in which the invention or idea has technical novelty, and is stated in a number of claims within the formal patent documentation. Patents protect the inventor's idea for a short length of time and allow the inventor time to develop and commercially exploit their ideas. During this time, no one else is able to use this idea without the patent-owner selling the rights. A patent remains in force for 20 years and elsewhere at least 20 years. 

\subsection{Trademark Protection}

Trade marks are a logo or symbol that can be licensed to third-parties. The main property which distinguishes it from copyright is that a trademark can be protected as long as it's in use. The trademark may not necessarily include an inventive step but does need to be unique and recognisable of the brand. Within the creative industries, for example, the film or book may be protected by copyright and the characters trademarked - this allows third party companies to 

If you're designing for pleasure or personal study, and don't intend to commercialise your ideas, then generally you don't need to worry about infringing other's IP (by this, I mean that there is an established legal defence [citation]). It's still important to be aware of the potential areas of patent or copyright infringment which, as examples, might include mechanisms for locking jars\sidenote[][-8\baselineskip]{Let's say, for example, that with your trusty dual extrusion FDM machine you designed you an all-in-one cylindrical enclosure with a clamping top onto a flexible gasket, intended for food and wanted to sell your design for others to buy. It's important to know what else has been protected. The Kilner jar is a famous example of sealing mechanism. The original jar was a screw-top. Later designs comprise a hinged lid and clamp which press the lid down on a deformable rubber seal. The clamp's downward pressure provides a seal enabling food to be preserved for longer. The design was patented in 1858. Kilner jars are, however, trademarked. http://www.mylearning.org/inventors-and-inventions-from-yorkshire/p-2619/}, or reproduction of film mechandise (although this is almost definitely a copyright infringement). 

% \subsection{Principles of...}

%New scissors patent
%http://www.boldip.com/bold-move-room-for-improvement-on-even-the-most-common-mechanical-devices-scissors/


%http://3dprintingindustry.com/news/many-3d-printing-patents-expiring-soon-heres-round-overview-21708/

%http://united-kingdom.taylorwessing.com/download/article_3d_printer_guide.html#.V3dj_e2VvCI
% \section{Intellectual Property Maneuvers: The Rise of Open Source Additive Manufacturing}
% 
% \subsection{1980s - Come on Eileen}
% 
% The first recognised 3D printing technique was Stereolithography, developed in 1984 by Charles Hull who later founded 3D Systems. This is discussed in more detail later on in this book. After patenting the technique, Charles Hull founded 3D Systems Corporation and started to commercialise this in the SLA-1 system in 1988. 
% 
% Selective laser sintering was also developed around this time (1987) by a Carl Deckard of the University of Texas. The patent for SLS was issued in 1989 and licensed to DTM Inc. which as a company was bought by 3D Systems. 
% 
% Stratasys was another rapid prototyping company around at the time. Whilst the Berlin wall fell in 1989, Scott Crump also developed the fused deposition modelling technique, the basis for much of the open source printing techniques. At the same time, EOS GmbH was also founded in Germany and became well-respected in the area of laser sintering.
% 
% \subsection{1990s - Nothing Compares 2 U}
% 
% Whereas the 1980s a few techniques were invented, in the 1990s these became more established and were joined by others. The FDM patent was awarded to Stratasys in 1992.  
% 
% Commercial Stereolithography machines started to become available in the 1990s, 
% 
% During the late twentieth and early twenty first century, with the development of the internet, software development became a collaborative movement. Software development moved from private enterprise developing closed source code repositories to the open source. This enabled software engineers to engage others who were similarly interested. 
% 
% Additive Manufacturing was developed alongside the development of the free software foundation, that later turned into the open source movement. However, as required by this intellectual protection, AM was developed solely by a small number of innovative companies. 
% 
% 3D Systems Inc. 
% 
% Stratasys
% 
% \subsection{2000s - Wannabe}
% 
% In the mid 2000s, the RepRap project came about, following funding of Adrian Bowyer at Bath University by the UK Engineering and Physical Sciences Research Council. 

\section{Open Source and Collaborative - a new way of working?}




Cathedral and the Bazaar 

\section{Open Source Licensing}
\label{licensing}

`A free software license is a notice that grants the recipient extensive rights to modify and redistribute that software. These actions are usually prohibited by copyright law, but the rights-holder (usually the author) of a piece of software can remove these restrictions by accompanying the software with a software license which grants the recipient these rights.'

Creative Commons (CC) is an organisation that provides a legal framework for the creative sharing of knowledge. The designer chooses the features of a CC license that they would like and then is free to share it on the appropriate website of their choice. There are important considerations which include
\begin{itemize}
 \item Would you like the design to be reused for commercial purposes? 
 \item Are derivative, modified variants of the designs allowed?
 \item Should derivative works be licensed under the same conditions?
\end{itemize}

Open to closed licensing
\begin{itemize}
 \item Public Domain: This mark or license allows the designer to say that this work is free from any form of copyright. Typically very old creative works are licensed in this way.  
 \item CC0: The CC0 mark is used with works that the designer would like no restrictions on its sharing, and that the works are younger than national copyright restrictions allow. 
 \item CC-BY: This license allows the user to redistribute the design for any purpose (even commercially), and to modify it in any way. Reproduction must also include attribution to the original designer and indicate how it was modified.  \url{https://creativecommons.org/licenses/by/2.0/}
 \item CC-BY-SA: This license follows the same conditions as CC-BY and the derivative work must also be under the same license. 
 \item CC-BY-ND: This license follows the same conditions as CC-BY and does not permit derivative works without further permission. 
 \item CC-BY-NC: This license follows the same conditions as CC-BY and permits any other usage except for commercial applications. 
 \item CC-BY-NC-SA: This license combines CC-BY with the same non-commercial and shearealike restrictions above. 
 \item CC-BY-NC-ND: This license combines CC-BY with the same non-commercial and no derivatives restrictions above. 
\end{itemize}

\section{Open Source Issues}

\section{Further Reading}

\begin{itemize}
 \item Fabricated: The New World of 3D Printing, by Hod Lipson and Melba Kurmann
\end{itemize}


% \end{document}
